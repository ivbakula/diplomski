\chapter{Sustavi s više stupnjeva slobode}
\section{Jednadžba gibanja slobodnih oscilacija}\label{slobodne_oscilacije}
Općenito, sustavi s više stupnjeva slobode modelirani su kao $N$ etažni posmični
okviri. Takavi sustavi se sastoje od $N$ koncentriranih masa, što znači da je potrebno 
pratiti $N$ različitih pomaka. Drugim riječima, jednadžba gibanja takvog sustava biti će
zadana kao sustav od N diferencijalnih jednadžbi drugog reda.
\par

Sustav s više stupnjeva slobode, koji će biti razmatran u ovom radu, je dvoetažni
posmični okvir prikazan na slijedećoj slici, a osnovni pojmovi biti će objašnjeni
pomoću slobodnih oscilacija navedenog modela.

\par
ubaci sliku sustava i ekvivalentnog modela
\par

Sustavi sa slike imaju dva dinamička stupnja slobode jer su moguće dvije translacije
dviju različitih masa, pa jednadžbu gibanja opisuje sustav od dvije diferencijelne
jednadžbe drugog reda.
\begin{equation}\label{eq:sustav_diferencijalnih}
    \begin{dcases}
        m\ddot{u}_1 + (k_1+k_2)u_1 -k_2u_2 = 0\\
        m\ddot{u}_2 - k_2u_1 + k_2 u_2 = 0
    \end{dcases}
\end{equation}

Zapisano u matričnoj formi:
\begin{equation}\label{eq:sustav_diferencijalnih_matricno}
    \begin{bmatrix}
        m_1 & 0 \\
        0   & m_2
    \end{bmatrix}
    \begin{Bmatrix}
        \ddot{u}_1\\
        \ddot{u}_2
    \end{Bmatrix}
    +
    \begin{bmatrix}
        k_1+k_2 & -k_2\\
        -k_2 & k_2
    \end{bmatrix}
    \begin{Bmatrix}
        u_1\\
        u_2
    \end{Bmatrix}
    =
    \begin{Bmatrix}
        0\\
        0
    \end{Bmatrix}
\end{equation}

Odnosno
\begin{equation}\label{eq:sustav_diferencijalnih_matricni_kratko}
    \mm\vtor{u}{:}+\kk\vtor{u}{}=\vtor{0}{}
\end{equation}
Iz \eqref{eq:sustav_diferencijalnih} i \eqref{eq:sustav_diferencijalnih_matricno}
vidi se da je sustav diferencijalnih jednadžbi povezan preko krutosti odnosno
matrice krutosti. Opći oblik rješenja sustava je slijedeći:
\begin{equation}\label{eq:rjesenje_opcenitog_sustava}
    \vtor{u(t)}{} = \vtor{\psi}{}q(t)
\end{equation}

Vektor $\psi$ možemo shvatiti kao konstantu integracije, a funkcija $q(t)$ je
jednostavna harmonijska funkcija oblika:
\begin{equation}\label{eq:jednostavna_harmonijska_funkcija}
    q(t)=A\cos(\omega t) + B\sin(\omega t)
\end{equation}
Druga derivacija \eqref{eq:jednostavna_harmonijska_funkcija} jest:
\begin{equation}\label{eq:jednostavna_harmonijska_funkcija_dd}
    \ddot{q}(t)=-\omega^2(
    \underbrace{
        A\cos(\omega t) + B\sin(\omega t)
    }_{\text{$q(t)$}}
    )
    =-\omega^2q(t)
\end{equation}

Stoga, druga derivacija od \eqref{eq:rjesenje_opcenitog_sustava} glasi:
\begin{equation}\label{eq:rjesenje_opcenitog_sustava_dd}
    \vtor{u}{:}=-\omega^2q(t)\vtor{\psi}{}
\end{equation}

Uvrštavanjem \eqref{eq:rjesenje_opcenitog_sustava} i \eqref{eq:rjesenje_opcenitog_sustava_dd}
u \eqref{eq:sustav_diferencijalnih_matricni_kratko} dobijemo:
\begin{equation}
    (\vtor{\psi}{}\kk-\omega^2\vtor{\psi}{}\mm)q(t)=\vtor{0}{}
\end{equation}
Prvo trivijalno rješenje je za $q(t)=0$ što implicira da je $u(t)=0$ (sustav
miruje). Netrivijalno rješenje se dobije izjednačavanjem zagrade s nulom:

\begin{equation}\label{eq:vlastite_vrijednosti}
    \kk\{\psi\}=\omega^2\mm\{\psi\}
\end{equation}
Izraz \eqref{eq:vlastite_vrijednosti} predstavlja realni problem vlastitih
vrijednosti odnosno matrični problem vlastitih vrijednosti. Potrebno je odrediti
dvije nepoznanice: 
\begin{enumerate}
    \item vlastite vektore $\psi$
    \item vlastite skalare $\omega^2$
\end{enumerate}

Prebacivanjem nepoznanica na jednu stranu dobijemo
\begin{equation}\label{eq:vlastite_vrijednosti_homogeno}
    (\kk-\omega^2\mm)\vtor{\psi}{}=\vtor{0}{}
\end{equation}

U općenitom slučaju, izraz \eqref{eq:vlastite_vrijednosti_homogeno} predstavlja
sustav od N algebarskih jednadžbi s N nepoznanica. Trivijalno rješenje sustava je za
$\{\psi\}={0}$, a netrivijalno se određuje raspisom determinante matrice
$[[k]-\omega^2[m]]$. Raspisom determinante navedene matrice, dobije se polinom N-tog
stupnja kojeg nazivamo \textit{karakterističnim polinomom}. Nultočke polinoma
predstavljaju vlastite vrijednosti $\omega^2$, odnosno kvadrirane prirodne
frekvencije. Da bi nultočke polinoma bile realne pozitivne vrijednosti, matrice $[m]$
i $[k]$ moraju biti simetrične i pozitivno definitne. Uvijeti za pozitivnu
definitnost u građevinarstvu su slijedeći:
\begin{enumerate}
    \item za matricu $\kk$ - broj i raspored ležajeva u ispravnoj mreži mora biti
        takav da se spriječe pomaci krutog tijela
    \item za matricu $\mm$ - moraju se ukloniti stupnjevi slobode bez pridružene
        koncentrirane mase. Uklanjanje nepotrebnih stupnjeva slobode, vrši se
        statičkom kondenzacijom.
\end{enumerate}

Vlastiti vektori $\psi$ se određuju uvrštavanjem vrijednosti $\omega^2$ u matricu
$[k-\omega^2m]$, stoga je očito da vektori $\psi$ nisu jedinstveni jer i njihovi
višektratnici zadovoljavaju jednadžbu \eqref{eq:vlastite_vrijednosti_homogeno}.
Vektori $\psi$ nazivaju se \textit{modalnim vektorima}, a definiraju oblik titranja
sustava na frekvenciji $\omega$. Prvi modalni vektor $\psi_1$, naziva se temeljnim
(osnovnim) modom, a frekvencija na kojoj sustav titra u navedenom modu ($\omega_1$)
naziva se \textit{vlastitom frekvencijom temeljnog moda}.
\par

Ako su sve prirodne frekvencije različite od nula i međusobno različite, tada su svi
vlastiti vektori linearno nezavisni. Skup od n linearno nezavisnih vektora čini bazu
n-dimenzionalnog vektorskog prostora, pa je ukupno rješenje sustava diferencijalnih
jednadžbi linearna kombinacija svih pojedinačnih rješenja.
\begin{equation}\label{eq:opce_rjesenje_sustava}
    \vtor{u(t)}{}=\sum_{n=1}^N\vtor{\psi}{}_nq_n 
\end{equation}
Pri čemu je $q_n$:
\begin{equation}
    q_n=A_n\cos(\omega_n t) + B_n\sin(\omega_n t)
\end{equation}

Raspisivanjem \eqref{eq:opce_rjesenje_sustava} dobivamo:
\[
	\begin{Bmatrix}
		u_1\\
		u_2\\
		\vdots\\
		u_n
	\end{Bmatrix}
	=
	q_1(t)
	\begin{Bmatrix}
		\psi_{1,1}\\
		\psi_{2,1}\\
		\vdots\\
		\psi_{N,1}
	\end{Bmatrix}
	+
	q_2(t)
	\begin{Bmatrix}
		\psi_{1,2}\\
		\psi_{2,2}\\
		\vdots\\
		\psi_{N,2}
	\end{Bmatrix}
		+
		\cdots
		+
	q_N(t)
	\begin{Bmatrix}
		\psi_{1,N}\\
		\psi_{2,N}\\
		\vdots\\
		\psi_{N,N}
	\end{Bmatrix}
	\]
\[
	=
	\begin{bmatrix}
		q_1(t) \psi_{1,1} + q_2(t) \psi_{1,2} + q_N(t) \psi_{1,N} \\
		q_1(t) \psi_{2,1} + q_2(t) \psi_{2,2} + q_N(t) \psi_{2,N} \\
		\vdots \\
		q_1(t) \psi_{N,1} + q_2(t) \psi_{N,2} + q_N(t) \psi_{N,N}
	\end{bmatrix}
	=
	\underbrace{
	\begin{bmatrix}
		\psi_{1,1} & \psi_{1,2} & \cdots & \psi_{1_N} \\
		\psi_{2,1} & \psi_{2,2} & \cdots & \psi_{2_N} \\
		\vdots & \vdots & \ddots & \vdots \\
		\psi_{N,1} & \psi_{N,2} & \cdots & \psi_{N,N} 
	\end{bmatrix}
	}_{\text{\large{$\ppsi$}}}
	\underbrace{
	\begin{Bmatrix}
		q_1(t) \\
		q_2(t) \\
		\vdots \\
		q_n(t)
	\end{Bmatrix}
	}_{\text{\large{$\overrightarrow{q}$}}}
\]

Matricu $\ppsi$ nazivamo modalna matrica, a komponente vektora $q$ nazivaju se
modalne koordinate. Opće rješenje pod \eqref{eq:opce_rjesenje_sustava} sada možemo 
zapisati matrično kao:
\begin{equation}
    \vtor{u(t)}{} = \ppsi \vtor{q}{}
\end{equation}


Osim modalne matrice postoji i spektralna matrica ($N$x$N$) koja se sastoji od N
svojstvenih vrijednosti $\omega^2$.
\[
	\oomega^2 
	= 
	\begin{bmatrix}
		\omega_1^2 & 0 & 0 & \cdots & 0 \\
		0 & \omega_2^2 & 0 & \cdots & 0 \\
		0 & 0 & \omega_3^2 & \cdots & 0 \\
		\vdots  & \vdots  & \vdots  & \ddots &  0 \\
		0 & 0 & 0 & \cdots &  \omega_N^2 
	\end{bmatrix}
\]


Za slučaj sustava s dva stupnja slobode, definiranog sustavom diferencijalnih
jednadžbi pod \eqref{eq:sustav_diferencijalnih_matricno}, prirodne frekvencije 
$\omega_1^2$ i $\omega_2^2$ dobivene su rješavanjem kvadratne jednadžbe karakterističnog
polinoma za $\omega^2$. Modalne vektore možemo zapisati kao:
\begin{align}
    \overrightarrow{\psi}_1=
    \begin{Bmatrix}
        \psi_1\\
        \psi_2
    \end{Bmatrix}
    &=
    \begin{Bmatrix}
        \ffrac{k_1+k_2-\omega_1^2m_1}{k_2}\\
        1
    \end{Bmatrix}\\
    \overrightarrow{\psi}_2=
    \begin{Bmatrix}
        \psi_1\\
        \psi_2
    \end{Bmatrix}
    &=
    \begin{Bmatrix}
        \ffrac{k_1+k_2-\omega_1^2m_1}{k_2}\\
        1
    \end{Bmatrix}
\end{align}

Ukupno rješenje sustava jest linearna kombinacija slijedećih vektora:
\begin{equation}
    \begin{dcases}
        \vtor{u}_1(t) = \vtor{\psi}_1 q_1(t) = \vtor{\psi}_1 (A_1\cos(\omega_1 t) + B_1\sin(\omega_1 t))\\
        \vtor{u}_2(t) = \vtor{\psi}_2 q_2(t) = \vtor{\psi}_2 (A_2\cos(\omega_2 t) + B_2\sin(\omega_2 t))
    \end{dcases}
\end{equation}

Stoga, ukupno opće rješenje glasi:
\begin{equation}\label{eq:ukupno_opce_rjesenje_2dof}
    \vtor{u}(t)=\vtor{u}_1(t)+\vtor{u}_2(t)
\end{equation}

Bitno je za napomenuti da modalni vektor $\psi$ ne određuje maksimalne iznose
ordinata već samo njihov relativni odnos, tj. modalni oblik ili oblik titranja. Da
bismo dobili amplitude $A_n$ i $B_n$, potrebno je rješiti inicijalni problem oblika 
$\vtor{u(0)}{} = \vtor{u}{}$ i $\vtor{u(0)}{.}=\vtor{u}{}$. Za slučaj slobodnog titranja,
konstante $A_n$ i $B_n$ glase:
\begin{align}
    A_n&=u_n(0)\\
    B_n&=\frac{\dot{u}_n(0)}{\omega_n}
\end{align}

Shematski prikaz modova sustava s dva stupnja slobode prikazan je na slijedećoj
slici.

\section{Ortogonalnost modova}
Kao što je već pokazano, modove (modalne oblike) definiraju vektori. Dva vektora su
međusobno ortogonalna (okomita) ukoliko je njihov skalarni produkt jednak nuli.
Razmotrimo li $r$-ti i $n$-ti mod sustava, dobijemo slijedeći sustav jednadžbi 
(iz \eqref{eq:sustav_diferencijalnih_matricni_kratko}):
\begin{equation}\label{eq:pocetni_sustav_ortogonalnost}
    \begin{dcases}
        (\kk-\omega_r^2\mm)\vtor{\psi}{}_r=\vtor{0}{}\\
        (\kk-\omega_n^2\mm)\vtor{\psi}{}_n=\vtor{0}{}
    \end{dcases}
\end{equation}

Donju jednadžbu pomnožimo s $\vtor{\psi}{}_r^T$. U gornjoj jednadžbi prvo
transponiramo $\vtor{\psi}{}_r$ te ju pomnožimo s $\vtor{\psi}{}_n$. Sustav jednadžbi
\eqref{eq:pocetni_sustav_ortogonalnost} postaje:
\begin{equation}\label{eq:konacni_sustav_ortogonalnost}
    \begin{dcases}
        \vtor{\psi}{}_r^T(\kk-\omega_r^2\mm)\vtor{\psi}{}_n=0\\
        \vtor{\psi}{}_r^T(\kk-\omega_n^2\mm)\vtor{\psi}{}_n=0
    \end{dcases}
\end{equation}

Oduzimanjem gornje i donje jednadžbe dobijemo:
\begin{equation}
    (\omega_n^2-\omega_r^2)\vtor{\psi}{}_r^T\mm\vtor{\psi}{}_s=0
\end{equation}

Za $\omega_n\neq\omega_r$ vrijedi:
\begin{equation}\label{eq:ortogonalnost_masa}
    \vtor{\psi}{}_r^T\mm\vtor{\psi}{}_n=0
\end{equation}

Uvrštavanjem $\vtor{\psi}{}_r^T\mm\vtor{\psi}{}_n=0$ u bilo koju od jednadžbi iz
\eqref{eq:konacni_sustav_ortogonalnost} dobijemo:
\begin{equation}\label{eq:ortogonalnost_krutost}
    \vtor{\psi}{}_r^T\kk\vtor{\psi}{}_n=0
\end{equation}

Jednadžbe pod \eqref{eq:ortogonalnost_masa} i \eqref{eq:ortogonalnost_krutost}
govore da su modovi, pomnoženi težinskim koeficijentima iz matrice krutosti ili
matrice masa, međusobno ortogonalni (valjda). Kažemo da su modovi međusobno ortogonalni s
obzirom na matricu mase ili matricu krutosti (valjda).
\par

Poslijedica ortogonalnosti je dijagonalnost slijedećih pravokutnih matrica:
\begin{alignat}{2}
    &\text{Modalna krutost}\quad & \mathbf{K}&=\ppsi^T\kk\ppsi\label{eq:modalna_krutost_matrica}\\
    &\text{Modalna masa}\quad &\mathbf{M}&=\ppsi^T\mm\ppsi\label{eq:modalna_masa_matrica}
\end{alignat}

Članovi matrica računaju se prema slijedećim formulama:
\begin{alignat}{2}
    &\text{Za modalnu krutost}\quad & K_{n,n}&=\vtor{\psi}{}_n^Tk\vtor{\psi}{}_n\label{eq:modalna_krutost}\\
    &\text{Za modalnu masu}\quad &M_{n,n}&=\vtor{\psi}{}_n^Tk\vtor{\psi}{}_n\label{eq:modalna_masa}
\end{alignat}

Između elemenata matrica vrijedi slijedeći odnos:
\begin{equation}
    \omega_n^2=\frac{K_n}{M_n}
\end{equation}
U matričnoj formi:
\begin{equation}
    \oomega^2=\mathbf{K}\mathbf{M}^{-1}
\end{equation}

\section{Normiranje modova}
Modalni vektori nisu jedinstveni jer su jednako predstavljeni vektorima dobivenim
rješenjem problema vlastitih vrijednosti i njihovim višekratnicima. Drugim riječima,
modalni vektor predstavljen je familijom kolinearnih vektora jer vrijedi slijedeća
jednakost (iz \eqref{eq:vlastite_vrijednosti})
\[
    \begin{aligned}
        \kk(a\vtor{\psi}{}_n)&=\omega_n^2(a\vtor{\psi}{}_n)\mm\\
        a\kk\vtor{\psi}{}_n&=a\omega_n^2\vtor{\psi}{}_n\mm\\
        \kk\vtor{\psi}{}_n&=\omega_n^2\vtor{\psi}{}_n\mm
    \end{aligned}
\]

Množenje vlastitog vektora skalarom, s ciljem postizanja željene forme modalnog
vektora, naziva se normiranje modalnog vektora odnosno moda. Primjerice, željena forma 
modalnog vektora može biti vektor čiji je najveći element jedan:
\[
    \vtor{\psi}{}=
        \begin{Bmatrix}
            \ffrac{1}{2}\\[8pt]
            \ffrac{3}{4}
        \end{Bmatrix}
\]
Množenjem vektora s $4/3$ dobijemo:
\[
    \vtor{\psi}{}=
        \begin{Bmatrix}
            \ffrac{2}{3}\\[6pt]
            1
        \end{Bmatrix}
\]

Od posebnog značaja je normiranje modalne mase na jediničnu vrijednost. Kako je
$\vtor{\psi}{}_n^T\mm\vtor{\psi}{}_n=\mathbf{M}_{n,n}$, mod $\vtor{\psi}{}_n$
potrebno je množiti sa $(M_{n,n})^{-1/2}$, odnosno:
\begin{equation}\label{eq:normiranje_masa}
    \vtor{\psi}{}_n^N=\frac{1}{\sqrt{M_{n,n}}}\vtor{\psi}{}_n
\end{equation}

Gdje je $\vtor{\psi}{}_n^N$ normirani vektor $n$-tog moda. Jednadžba
\eqref{eq:normiranje_masa} zapisana u matričnoj formi glasi:
\begin{equation}\label{eq:normiranje_masa_matricno}
    \ppsi_n^N=\ppsi\,\mathbf{M}^{-\frac{1}{2}}
\end{equation}

Vrijedi slijedeća relacija:
\begin{equation}\label{eq:relacija_normirani}
        M_{n,n}=\left(\vtor{\psi}{}_n^N\right)^T\mm\vtor{\psi}{}_n^N=1
\end{equation}

Odnosno u matričnom obliku:
\begin{equation}\label{eq:relacija_normirani_matricno}
    \left(\ppsi^N\right)^T\mm\ppsi^N=I
\end{equation}

Gdje je $I$ jedinična matrica. Iz \eqref{eq:modalna_krutost} slijedi:
\begin{equation}
    \begin{split}
        \left(\ppsi^N\right)^T\K\ppsi^N=\oomega^2\underbrace{\left(\ppsi^N\right)^T\M\ppsi^N}_{\text{$I$}} \notag\\
        \left(\ppsi^N\right)^T\K\ppsi^N=\oomega^2\label{eq:normirana_krutost}
    \end{split}
\end{equation}

\section{Odziv sustava s više stupnjeva slobode na pobudu sinusnom
silom}\label{mdof_prisilne}
Kao što je pokazano u poglavlju \ref{slobodne_oscilacije}, jednadžba gibanja sustava
s $N$ stupnjeva slobode zadana je kao sustav od $N$ diferencijalnih jednadži drugog
reda. U slučaju pobude harmonijskom silom, navedeni sustav će se sastojati od $N$
nehomogenih diferencijalnih jednadžbi drugog reda, koje će biti povezane preko
matrice krutosti i/ili matrice mase. 
\par
Općenito, rješenje jedne proizvoljne nehomogene diferencijalne jednadžbe drugog reda
oblika $\alpha\ddot{y}+\beta\dot{y}+\gamma y= f(t)$ jest suma komplementarnog
rješenja $y_c$ i partikularnog rješenja $Y_p$.
\begin{equation}
    y(t)=y_c(t)+U_p(t)
\end{equation}

Komplementarno rješenje dobijemo izjednačavanjem diferencijalne jednadžbe s nulom
odnosno:
\begin{equation}\label{eq:opce_komplementarno_rjesenje}
    \alpha\ddot{y}+\beta\dot{y}+\gamma y=0
\end{equation}

Primjetimo da je komplementarno rješenje (rješenje jednadžbe \eqref{eq:opce_komplementarno_rjesenje}) 
zapravo rješenje homogene diferencijalne jednadžbe, a jednako je i za slobodne 
oscilacije i za prisilne oscilacije. Kod prisilnih oscilacija, komplementarno 
rješenje predstavlja prolazni dio odziva. Partikularno riješenje možemo pronaći 
koristeći se metodom neodređenih koeficijenata, a predstavlja prolazni dio odziva.
\par

Analogno tome, komplementarno rješenje sustava diferencijalnih jednadžbi dato je u
\eqref{eq:opce_rjesenje_sustava} te predstavlja prolazni dio odziva, pa u slučaju
prisilnih oscilacija preostaje nam odrediti još partikularno rješenje koje
predstavlja prisilni dio odziva. 
\par

Zadana je jednadžba gibanja sustava s dva stupnja slobode prikazanog na slijedećoj
slici

\textbf{UBACI SLIKU SUSTAVA}

\begin{equation}\label{eq:jednadzba_gibanja_matricno}
    \begin{bmatrix}
        m_1 & 0 \\
        0   & m_2
    \end{bmatrix}
    \begin{Bmatrix}
        \ddot{u}_1\\
        \ddot{u}_2
    \end{Bmatrix}
    +
    \begin{bmatrix}
        k_1+k_2 & -k_2\\
        -k_2 & k_2
    \end{bmatrix}
    \begin{Bmatrix}
        u_1\\
        u_2
    \end{Bmatrix}
    =
    \begin{Bmatrix}
        p_0\\
        0 
    \end{Bmatrix}
    \sin(\omega t)
\end{equation}

Odnosno:
\begin{equation}\label{eq:jednadzba_gibanja_matricno_kratko}
    \mm\vtor{u}{:}+k\vtor{u}{}=\vtor{p_n}{}\sin(\omega t)
\end{equation}

Gdje je $\vtor{p_n}{}$ vektor amplituda harmonijskih sila. Odziv sustava biti će
harmonijski, jednake frekvencije, pa partikularno rješenje možemo pretpostaviti:
\begin{equation}\label{eq:partikularno_pretpostavka}
    \vtor{u_p(t)}{}=\vtor{U_n}{}\sin(\omega t)
\end{equation}
Gdje je $\vtor{U_n}{}$ vektor koeficijenata.
Druga derivacija \eqref{eq:partikularno_pretpostavka} glasi:
\begin{equation}\label{eq:partikularno_pretpostavka_dd}
    \{\ddot{u}(t)\}=-\omega^2\vtor{U_n}{}\sin(\omega t)
\end{equation}

Uvrštavanjem \eqref{eq:partikularno_pretpostavka} i \eqref{eq:partikularno_pretpostavka_dd}
u \eqref{eq:jednadzba_gibanja_matricno_kratko} dobijemo:
\begin{equation}\label{eq:neodredjeni_koeficijenti_1}
    -\omega^2\vtor{U_n}{}\mm\sin(\omega t) 
    +
    \vtor{U_n}{}\kk\sin(\omega t)
    =
    \vtor{p}{}\sin(\omega t)
\end{equation}

Nakon sređivanja, jednadžba \eqref{eq:neodredjeni_koeficijenti_1} poprima oblik:
\begin{equation}\label{eq:neodredjeni_koeficijenti_2}
    [\kk-\omega^2\mm]\vtor{U_n}{}=\vtor{p}{}
\end{equation}

%Matrica $[\kk-\omega^2\mm]$ predstavlja matricu dinamičke krutosti sustava s više stupnjeva
%slobode, a označava se kao $[Z(\omega)]$. Jednadžbu pod \eqref{eq:neodredjeni_koeficijenti_2}
%možemo zapisati kao:
%\begin{equation}\label{eq:neodredjeni_koeficijenti_z}
%    [Z(\omega)]\vtor{U_n}{}=\vtor{p_n}{}
%\end{equation}
%
%Konačno, vektor koeficijenata dobijemo množenjem izraza \eqref{eq:neodredjeni_koeficijenti_z}
%inversom matrice dinamičke krutosti, te dobijemo:
%\begin{equation}
%    \vtor{U_n}{}=\vtor{p_n}{}\,[Z(\omega)]^{-1}
%\end{equation}

Množenjem jednadžbe \eqref{eq:neodredjeni_koeficijenti_2} s $[\kk-\omega^2\mm]^{-1}$
dobijemo:
\begin{equation}
    \begin{split}
        \vtor{U_n}{}=[\kk-\omega^2\mm]^{-1}\vtor{p_n}{} \notag\\
        \vtor{U_n}{}=\frac{1}{det[\kk-\omega^2\mm]}adj[\kk-\omega^2\mm]\vtor{p_n}{}\label{eqneodredjeni_koeficijenti_3}
    \end{split}
\end{equation}

Odnosno u matričnom obliku:
\begin{equation}
    \begin{Bmatrix}
        U_1\\
        U_2
    \end{Bmatrix}
    =
    \frac{1}{det[\kk-\omega^2\mm]}
        \begin{bmatrix}
            k_2-m_2\omega^2 & k_2\\
            k_2 & k_1+k_2-m_1\omega^2
        \end{bmatrix}
        \begin{Bmatrix}
            p_0\\
            0
        \end{Bmatrix}
\end{equation}

Komponente vektora $\vtor{U_n}{}$ glase:
\begin{align}
    U_1=\frac{p_0(k_2-m_2\omega^2)}{m_1m_2(\omega^2-\omega_1^2)(\omega^2-\omega_2^2)} \label{eq:vtor_1_opce}\\
    U_2=\frac{p_0k_2}{m_1m_2(\omega^2-\omega_1^2)(\omega^2-\omega_2^2)}\label{eq:vtor_2_opce}
\end{align}

Za $m_1=2m$, $m_2=m$, $k_1=2k$ i $k_2=k$ vektori glase:
\begin{align}
    U_1=\frac{p_0(k-m\omega^2)}{2m^2(\omega^2-\omega_1^2)(\omega^2-\omega_2^2)}\label{eq:vtor_1}\\
    U_2=\frac{p_0k}{2m^2(\omega^2-\omega_1^2)(\omega^2-\omega_2^2)}\label{eq:vtor_2}
\end{align}

Uz $\omega_1=\sqrt{k/2m}$ i $\omega_2=\sqrt{2k/m}$ te dijeljenjem \eqref{eq:vtor_1} i \eqref{eq:vtor_2}
dobijemo vektor dinamičkog koeficijenta pomaka (bez dimenzija), koji ovisi o
omjerima frekvencija $\omega/\omega_1$ te $\omega/\omega_2$. Vektor je prikazan u nastavku
\begin{equation}
    \frac{2k}{p_0}\vtor{U}{}
    =
    \begin{Bmatrix}
        \ffrac{1-0.5(\omega/\omega_1)^2}
              {[1-(\omega/\omega_1)^2][1-(\omega/\omega_2)^2]}\\[12pt]
        \ffrac{1}
              {[1-(\omega/\omega_1)^2][1-(\omega/\omega_2)^2]}
    \end{Bmatrix}
\end{equation}

Prisilni dio odziva glasi:
\begin{equation}
    \vtor{u(t)}{} = \vtor{U}{}\sin(\omega t) = 
    \begin{Bmatrix}
        \ffrac{1-0.5(\omega/\omega_1)^2}
              {[1-(\omega/\omega_1)^2][1-(\omega/\omega_2)^2]}\\[12pt]
        \ffrac{1}
              {[1-(\omega/\omega_1)^2][1-(\omega/\omega_2)^2]}
    \end{Bmatrix}
    \sin(\omega t)
\end{equation}

Komponente vektora $\vtor{U}{}$ možemo iscrtati kao graf funkcije dinamičkog faktora
$U_1/p_0/2k$ i $U_2/p_0/2k$ u ovisnosti o frekvencijskom omjeru $\omega/\omega_1$.

\section{Interesantan postupak za rješavanje 2dof}
Zadana je jednadžba gibanja prisilnog titranja sustava s dva stupnja slobode.
\begin{equation}\label{eq:jdn_gibanja_osnovno}
    \mm\vtor{u}{:}+\kk\vtor{u}{}=\vtor{p}{}\sin(\omega t)
\end{equation}

Odziv sustava će biti sinusni, iste frekvencije kao i pobuda, pa rješenje
pretpostavljamo u slijedećem obliku:
\begin{equation}\label{eq:pretpostavljeno}
    \vtor{U_p}{}=\vtor{U}{}\sin(\omega t)
\end{equation}

Druga derivacija pretpostavljenog rješenja glasi:
\begin{equation}\label{eq:pretpostavljeno_dd}
    \vtor{U_p}{}=-\omega^2\vtor{U}{}\sin(\omega t)
\end{equation}

Uvrštavanjem \eqref{eq:pretpostavljeno} i \eqref{eq:pretpostavljeno_dd} u
\eqref{eq:jdn_gibanja_osnovno} dobijemo:
\begin{equation}
    -\omega^2\vtor{U}{}\mm\sin(\omega t) + \mm\vtor{U}{}\sin(\omega t) = \vtor{p}{}\sin(\omega t)
\end{equation}

Te nakon sređivanja:
\begin{equation}\label{eq:jdn_gibanja_part_1}
    [\kk-\omega^2\mm]\vtor{U}{}=\vtor{p}{}
\end{equation}

Matrica $[\kk-\omega^2\mm]$ predstavlja matricu dinamičke krutosti.
\begin{equation}\label{eq:mat_din_krutost}
    [\kk-\omega^2\mm]=\mathbf{Z}
\end{equation}

Da bismo rješili \eqref{eq:jdn_gibanja_part_1} potrebno je pronaći invers matrice
dinamičke krutosti. Invers matrice dinamičke krutosti predstavlja matrica
frekvencijskih funkcija odziva, a označavamo ju s $\mathbf{H}$. Jednadžba pod 
\eqref{eq:mat_din_krutost} može poprimiti slijedeći oblik:
\begin{equation}
    \left(\ppsi^N\right)^T[\kk-\omega^2\mm]\ppsi^N
    =
    \left(\ppsi^N\right)^T\mathbf{Z}\;\ppsi^T
\end{equation}
Odnosno
\begin{equation}
     \left(\ppsi^N\right)^T[\kk-\omega^2\mm]\ppsi^N
    =
    \left(\ppsi^N\right)^T\mathbf{H}^{-1}\;\ppsi^T
\end{equation}

Iz \eqref{eq:relacija_normirani_matricno} i \eqref{eq:normirana_krutost} slijedi:
\begin{equation}
    [\omega_n^2 - \omega^2] = \left(\ppsi^N\right)^T\mathbf{H}^{-1}\;\ppsi^T
\end{equation}

Nakon sređivanja, dobijemo:
\begin{equation}
    \mathbf{H}=\left(\ppsi^N\right)^T[\omega_n^2 - \omega^2]^{-1}\;\ppsi^T
\end{equation}

Član $H_{j,k}$ matrice $\mathbf{H}$ možemo dobiti preko slijedeće jednadžbe:
\begin{equation}\label{eq:clan_frf_matrice}
    H_{j,k}=\frac{\psi^N_{j,1}\psi^N_{k,1}}{\omega_1^2-\omega^2}
            +
            \frac{\psi^N_{j,2}\psi^N_{k,2}}{\omega_2^2-\omega^2}
            +
            \frac{\psi^N_{j,3}\psi^N_{k,3}}{\omega_3^2-\omega^2}
            +
            \cdots
            +
            \frac{\psi^N_{j,n}\psi^N_{k,n}}{\omega_n^2-\omega^2}
\end{equation}

Jednadžbu \eqref{eq:clan_frf_matrice} možemo zapisati kao skalarni umnožak vektora:
\begin{equation}\label{eq:clan_frf_matrica_vektorski}
    H_{j,k}
    =
    \begin{Bmatrix}
        \psi_{j,1}\psi_{k,1} &
        \psi_{j,2}\psi_{k,2} &
        \psi_{j,3}\psi_{k,3} &
        \cdots
        \psi_{j,n}\psi_{k,n}
    \end{Bmatrix}
    \begin{Bmatrix}
        \ffrac{1}{\omega_1^2-\omega^2}\\[6pt]
        \ffrac{1}{\omega_2^2-\omega^2}\\[6pt]
        \ffrac{1}{\omega_3^2-\omega^2}\\[6pt]
        \vdots\\[6pt]
        \ffrac{1}{\omega_n^2-\omega^2}
    \end{Bmatrix}
\end{equation}

Za zadani sustav, matrica modova normiranih s obzirom na masu $\ppsi^N$ glasi:
\[
    \ppsi^N
    =
    \begin{Bmatrix}
        \ffrac{\sqrt{6m}}{6m} & -\ffrac{\sqrt{3m}}{3m}\\[12pt]
        \ffrac{\sqrt{6m}}{6m} & \ffrac{\sqrt{3m}}{3m}
    \end{Bmatrix}
\]

Izračun matrice frekvencijske funkcije odziva $H_{1,1}$ prikazan je u nastavku:
\[
\begin{aligned}
    H_{1,1} &=
    \frac{\psi^N_{1,1}}{\omega_1^2-\omega^2}+\frac{\psi^N_{1,1}}{\omega_2^2-\omega^2}
    =
    \frac{\ffrac{\sqrt{6m}}{6m}\ffrac{\sqrt{6m}}{6m}}{\omega_1^2-\omega^2}
    +
    \frac{\ffrac{\sqrt{3m}}{3m}\ffrac{\sqrt{3m}}{3m}}{\omega_2^2-\omega^2}\\
    %
    H_{1,1} &=
    \frac{1}{3m} 
        \left(
            \frac{\ffrac{1}{2}}{(\omega_1^2-\omega^2}
            +
            \frac{1}{(\omega_2^2-\omega^2)}
        \right)
    =
    \frac{1}{3m}
        \left(
            \frac{\ffrac{1}{2}(\omega_2^2-\omega^2)+(\omega_1^2-\omega^2)}
                {\omega_1^2\omega_2^2(1-(\omega/\omega_1)^2)(1-(\omega/\omega_2)^2}
        \right)\\
\end{aligned}
\]

U brojniku, $\omega_2$ izrazimo kao $\omega_1$ pomoću relacije $\omega_2=\sqrt{2k/m}$
i $\omega_1=\sqrt{k/2m}$.
\[
    \begin{aligned}
        H_{1,1}=
        \frac{1}{3m}
            \left(
                \frac{3-\ffrac{3}{2}(\omega/\omega_1)^2}
                    {\omega_2^2(1-(\omega/\omega_1)^2)(1-(\omega/\omega_2)^2)}
            \right)
    \end{aligned}
\]
Nakon sređivanja:
\begin{equation}\label{eq:frf_11}
    H_{1,1}=\frac{1-\ffrac{1}{2}(\omega/\omega_1)^2}{2k(1-(\omega_1/\omega)^2)(1-(\omega_2/\omega)^2)}
\end{equation}

Analogno tome, dobiju se ostali članovi matrice te glase:
\begin{align}
    H_{1,2} &= \frac{1}{2k(1-(\omega_1/\omega)^2)(1-(\omega_2/\omega)^2)}\label{eq:frf_12}\\
    H_{2,1} &= \frac{1}{2k(1-(\omega_1/\omega)^2)(1-(\omega_2/\omega)^2)}\label{eq:frf_21}\\
    H_{2,2} &= \frac{1-2(\omega/\omega_2)^2}{2k(1-(\omega_1/\omega)^2)(1-(\omega_2/\omega)^2)}\label{eq:frf_22}\\
\end{align}

Matrica $\mathbf{H}$ glasi:
\begin{equation}\label{eq:matrica_frf}
    \mathbf{H}=\frac{1}{2k(1-(\omega/\omega_1)^2(1-(\omega/\omega_2)^2)}
    \begin{bmatrix}
        1-\ffrac{1}{2}\left(\ffrac{\omega}{\omega_1}\right)^2 & 1 \\
        1 & 1-2\left(\ffrac{\omega}{\omega_2}\right)^2
    \end{bmatrix}
\end{equation}

Uvrštavanjem \eqref{eq:matrica_frf} u \eqref{eq:jdn_gibanja_part_1} dobijemo:
\begin{equation}
    \mathbf{H}^{-1}\vtor{U}{}=\vtor{p}{}
\end{equation}

Množenjem prethodnog izraza s $\mathbf{H}$:
\begin{equation}
    \vtor{U}{} = \mathbf{H}\vtor{p}{}
\end{equation}

Raspisivanjem prethodne jednadžbe:
\begin{equation}
    \begin{Bmatrix}
        U_1\\
        U_2
    \end{Bmatrix}
    =
    \frac{1}{2k\left(1-\left(\ffrac{\omega_1}{\omega}\right)^2\right)\left(1-\left(\ffrac{\omega_2}{\omega}\right)^2\right)}
    %
    \begin{bmatrix}
        1-\ffrac{1}{2}\left(\ffrac{\omega}{\omega_1}\right)^2 & 1 \\
        1 & 1-2\left(\ffrac{\omega}{\omega_2}\right)^2
    \end{bmatrix}
    % 
    \begin{Bmatrix}
        p_0\\
        0
    \end{Bmatrix}
\end{equation}

Te konačno, vektor $\vtor{U}{}$ glasi:
\begin{equation}\label{eq:vektor_u_konacno}
    \begin{Bmatrix}
        U_1\\
        U_2
    \end{Bmatrix}
    =
    \frac{1}{2k\left(1-\left(\ffrac{\omega_1}{\omega}\right)^2\right)\left(1-\left(\ffrac{\omega_2}{\omega}\right)^2\right)}
    \begin{Bmatrix}
        p_0\left(1-\ffrac{1}{2}\left(\ffrac{\omega}{\omega_1}\right)^2\right)\\
        p_0 
    \end{Bmatrix}
\end{equation}

Dijeljenjem vektora $\vtor{U}{}$ s $p_0/2k$ dobijemo slijedeće:
\begin{equation}
    \begin{Bmatrix}
        U_1\\
        U_2
    \end{Bmatrix}
    =
    \frac{1}{\left(1-\left(\ffrac{\omega_1}{\omega}\right)^2\right)\left(1-\left(\ffrac{\omega_2}{\omega}\right)^2\right)}
    \begin{Bmatrix}
        1-\ffrac{1}{2}\left(\ffrac{\omega}{\omega_1}\right)^2\\
        1 
    \end{Bmatrix}
\end{equation}

%Komponente vektora zapisane posebno:
%\begin{align}
%    U_1&=\frac{1-\ffrac{1}{2}\left(\ffrac{\omega}{\omega_1}\right)^2}
%        {\left(1-\left(\ffrac{\omega_1}{\omega}\right)^2\right)\left(1-\left(\ffrac{\omega_2}{\omega}\right)^2\right)}\\
%    U_2&=\frac{1}{\left(1-\left(\ffrac{\omega_1}{\omega}\right)^2\right)\left(1-\left(\ffrac{\omega_2}{\omega}\right)^2\right)}
%\end{align}

Komponente vektora $U$ predstavljaju funkcije ovisnosti dinamičkog faktora o 
frekvencijskim omjerima $\omega/\omega_1$ i $\omega/\omega_2$. Njihovi grafovi
prikazani su u nastavku.

\textbf{UBACI GRAFOVE}

Prisilni dio odziva glasi:
\begin{equation}
    \begin{split}
        \vtor{u(t)}{} = \vtor{U}{}\; sin(\omega t)
        %\vtor{u(t)}{}=
        =
        \frac{1}{\left(1-\left(\ffrac{\omega_1}{\omega}\right)^2\right)\left(1-\left(\ffrac{\omega_2}{\omega}\right)^2\right)}
    \begin{Bmatrix}
        1-\ffrac{1}{2}\left(\ffrac{\omega}{\omega_1}\right)^2\\
        1 
    \end{Bmatrix}
    \sin(\omega t)
    \end{split}
\end{equation}

\section{Modalna analiza}
Modalna analiza je postupak određivanja osnovnih dinamičkih parametara linearnog
sustava s ciljem definiranja matematičkog modela ponašanja sustava pod utjecajem
dinamičkih sila. Modalna analiza temelji se na \textit{principu superpozicije},
odnosno na činjenici da se ukupni odziv sustava može zapisati kao linearna
kombinacija odziva pojedinih modova.
\par

Razmotrimo jednadžbu gibanja sustava s više stupnjeva slobode pobuđenog proizvoljnom
silom:
\begin{equation}\label{eq:mdof_jdn_gibanja_opca_pobuda}
    \mm\vtor{u}{:} + \kk\vtor{u}{} = \vtor{p(t)}{}
\end{equation}

Klasično rješenje jednadžbe gibanja \eqref{eq:mdof_jdn_gibanja_opca_pobuda}
prikazano je u poglavlju ~\ref{mdof_prisilne} na primjeru sustava s dva stupnja 
slobode pobuđenog harmonijskom silom. Za sustave s više od dva stupnja slobode ili
za složenije sile pobude, rješavanje jednadžbe gibanja na klasični način može biti
izuzetno teško ili nemoguće. U takvim slučajevima, jednadžba gibanja se rješava
postupcima modalne analize.
\par

Iz poglavlja ~\ref{slobodne_oscilacije} znamo da je rješenje jednadžbe gibanja
slobodnog titranja sustava s više stupnjeva slobode linearna kombinacija
odziva svih pojedinih modova odnosno:
\begin{equation}\label{eq:mdf_ukupno_rjesenje}
    u(t)=\sum_{r=1}^N\mm\psi_rq_r(t)
\end{equation}

Uvrštavanjem \eqref{eq:mdf_ukupno_rjesenje} u \eqref{eq:mdof_jdn_gibanja_opca_pobuda} 
dobijemo:
\begin{equation}\label{eq:mdf_modalna_1}
    \sum_{r=1}^N\mm\psi_r\ddot{q}_r(t) + \sum_{r=1}^N\kk\psi_rq_r(t)=p(t)
\end{equation}

Množenjem jednadžbe \eqref{eq:mdf_modalna_1} s $\psi_n^T$ dobijemo:
\begin{equation}\label{eq:mdf_modalna_2}
    \sum_{r=1}^N\psi_n^T\mm\psi_r\ddot{q}_r(t)+\sum_{r=1}^N\psi_n^T\kk\psi_rq_r(t)=\psi_n^Tp(t)
\end{equation}

Zbog svojstva ortogonalnosti, isčezavaju svi članovi sumacija osim $n$-tog člana, pa
preostaje:
\begin{equation}\label{eq:mdf_modalna_jednadzba_nesredjeno}
    \psi_n^T\,\mm\,\psi_n\,\ddot{q}_n(t)\;+\;\psi_n^T\,\kk\,\psi_n\,q_r(t)=\psi_n^T\,p(t)
\end{equation}

Koristeći relacije iz \eqref{eq:modalna_masa} i \eqref{eq:modalna_krutost} jednadžba
\eqref{eq:mdf_modalna_jednadzba_nesredjeno} poprima slijedeći oblik:
\begin{equation}\label{eq:mdf_modalna_jednadzba_sredjeno}
    M_n\ddot{q}_n(t)+K_nq_n(t)=P_n(t)
\end{equation}

Gdje su $M_n$, $K_n$ i $P_n$ poopćena masa, krutost i opterećenje $n$-tog moda.

Postupkom, prikazanim u jednadžbama \eqref{eq:mdf_ukupno_rjesenje},
\eqref{eq:mdf_modalna_1}, \eqref{eq:mdf_modalna_2},
\eqref{eq:mdf_modalna_jednadzba_nesredjeno}, jednadžbu gibanja sustava predstavljenu
sustavom diferencijalnih jednadžbi sveli smo na skup međusobno neovisnih
diferencijalnih jednadžbi. Drugim riječima, sustav od $N$ stupnjeva slobode razložen
je na $N$ međusobno neovisnih podsustava s jednim stupnjem slobode (princip superpozicije), 
pri čemu $n$-ti podsustav prikazuje odziv sustava u $n$-tom modu. Podsustave 
nazivamo \textit{poopćeni sustav za n-ti mod}. "Jednadžba gibanja" 
poopćenog sustava za $n$-ti oblik titranja predstavljena je diferencijalnom jednadžbom
\eqref{eq:mdf_modalna_jednadzba_sredjeno} čije rješenje predstavlja modalnu 
koordinatu $n$-tog moda $q_n(t)$.
\par

Odziv $n$-tog moda je:
\begin{equation}\label{eq:mdf_odziv_ntog_moda}
    u_n(t)=\psi_nq_n(t)
\end{equation}

Da bismo odredili ukupni odziv sustava s $N$-stupnjeva slobode, potrebno je riješiti
$N$ modalnih jednadžbi, oblika definiranog pod \eqref{eq:mdf_modalna_jednadzba_sredjeno}.
Matrični zapis sustava modalnih jednadžbi prikazan je u nastavku.
\begin{equation}\label{eq:sustav_modalnih_jednadzbi}
    \M\vtor{q}{:} + \K\vtor{q}{}=\vtor{P(t)}{}
\end{equation}

Gdje je $\M$ matrica modalnih masa, $\K$ matrica modalnih krutosti, $\vtor{P(t)}{}$
vektor poopćenih opterećenja. Iz \eqref{eq:modalna_masa_matrica} i
\eqref{eq:modalna_krutost_matrica} znamo da su matrice $\M$ i $\K$ dijagonalne što
znači da je \eqref{eq:sustav_modalnih_jednadzbi} sustav međusobno neovisnih jednadžbi.
Rješenjem navedenog sustava, dobijemo funkcije modalnih koordinata za sve modove
sustava, a ukupni odziv definirano je linearnom kombinacijom (princip superpozicije) 
odziva svakog pojedinog moda. Odziv pojedinog moda definiran je
\eqref{eq:mdf_odziv_ntog_moda}, a ukupni odziv je:

\begin{equation}\label{eq:mdf_modalna_ukupno_rjesenje}
    u(t)=\sum_{n=1}^Nu_n(t)=\sum_{n=1}^N\psi_nq_n(t)
\end{equation}
