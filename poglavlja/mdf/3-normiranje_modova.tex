\section{Normiranje vlastitih vektora}\label{sec:normiranje}
Vlastiti vektori nisu jednoznačni jer su jednako predstavljeni vektorima dobivenim
rješenjem problema vlastitih vrijednosti i njihovim višekratnicima. Drugim riječima,
vlastiti vektor predstavljen je familijom kolinearnih vektora jer vrijedi slijedeća
jednakost (~\cite{hefangfu2001})(iz \eqref{eq:vlastite_vrijednosti})
\[
    \begin{aligned}
        \kk(a\vtor{\psi}{}_n)&=\omega_n^2(a\vtor{\psi}{}_n)\mm\\
        a\kk\vtor{\psi}{}_n&=a\omega_n^2\vtor{\psi}{}_n\mm\\
        \kk\vtor{\psi}{}_n&=\omega_n^2\vtor{\psi}{}_n\mm
    \end{aligned}
\]

Množenje vlastitog vektora skalarom, s ciljem postizanja željenog oblika vlastitog
vektora, naziva se normiranje. Primjerice, željeni oblik modalnog vektora može biti 
vektor čiji je najveći element jedan:
\[
    \vtor{\psi}{}=
        \begin{Bmatrix}
            \ffrac{1}{2}\\[8pt]
            \ffrac{3}{4}
        \end{Bmatrix}
\]
Množenjem vektora s $4/3$ dobijemo:
\[
    \vtor{\psi}{}=
        \begin{Bmatrix}
            \ffrac{2}{3}\\[6pt]
            1
        \end{Bmatrix}
\]

Od posebnog značaja je normiranje modalne mase na jediničnu vrijednost. Kako je
$\vtor{\psi}{}_n^T\mm\vtor{\psi}{}_n=\mathbf{M}_{n,n}$, vlastiti vektor $\vtor{\psi}{}_n$
potrebno je množiti sa $(M_{n,n})^{-1/2}$, odnosno:
\begin{equation}\label{eq:normiranje_masa}
    \vtor{\psi}{}_n^N=\frac{1}{\sqrt{M_{n,n}}}\vtor{\psi}{}_n
\end{equation}

Gdje je $\vtor{\psi}{}_n^N$ normirani $n$-ti vlastiti vektor. Jednadžba
\eqref{eq:normiranje_masa} zapisana u matričnoj formi glasi:
\begin{equation}\label{eq:normiranje_masa_matricno}
    \ppsi_n^N=\ppsi\,\mathbf{M}^{-\frac{1}{2}}
\end{equation}

Normiranje na jediničnu vrijednost jest u biti:
\begin{equation}\label{eq:relacija_normirani}
        M_{n,n}=\left(\vtor{\psi}{}_n^N\right)^T\mm\vtor{\psi}{}_n^N=1
\end{equation}

Odnosno u matričnom obliku:
\begin{equation}\label{eq:relacija_normirani_matricno}
    \left(\ppsi^N\right)^T\mm\ppsi^N=\text{I}
\end{equation}

Gdje je I jedinična matrica. Iz \eqref{eq:modalna_krutost} slijedi:
\begin{equation}
    \begin{split}
        \left(\ppsi^N\right)^T\K\ppsi^N=\oomega^2\underbrace{\left(\ppsi^N\right)^T\M\ppsi^N}_{\text{I}} \notag\\
        \left(\ppsi^N\right)^T\K\ppsi^N=\oomega^2\label{eq:normirana_krutost}
    \end{split}
\end{equation}


