\section{Interesantan postupak za rješavanje 2dof}
Zadana je jednadžba gibanja prisilnog titranja sustava s dva stupnja slobode.
\begin{equation}\label{eq:jdn_gibanja_osnovno}
    \mm\vtor{u}{:}+\kk\vtor{u}{}=\vtor{p}{}\sin(\omega t)
\end{equation}

Odziv sustava će biti sinusni, iste frekvencije kao i pobuda, pa rješenje
pretpostavljamo u slijedećem obliku:
\begin{equation}\label{eq:pretpostavljeno}
    \vtor{U_p}{}=\vtor{U}{}\sin(\omega t)
\end{equation}

Druga derivacija pretpostavljenog rješenja glasi:
\begin{equation}\label{eq:pretpostavljeno_dd}
    \vtor{U_p}{}=-\omega^2\vtor{U}{}\sin(\omega t)
\end{equation}

Uvrštavanjem \eqref{eq:pretpostavljeno} i \eqref{eq:pretpostavljeno_dd} u
\eqref{eq:jdn_gibanja_osnovno} dobijemo:
\begin{equation}
    -\omega^2\vtor{U}{}\mm\sin(\omega t) + \mm\vtor{U}{}\sin(\omega t) = \vtor{p}{}\sin(\omega t)
\end{equation}

Te nakon sređivanja:
\begin{equation}\label{eq:jdn_gibanja_part_1}
    [\kk-\omega^2\mm]\vtor{U}{}=\vtor{p}{}
\end{equation}

Matrica $[\kk-\omega^2\mm]$ predstavlja matricu dinamičke krutosti.
\begin{equation}\label{eq:mat_din_krutost}
    [\kk-\omega^2\mm]=\mathbf{Z}
\end{equation}

Da bismo rješili \eqref{eq:jdn_gibanja_part_1} potrebno je pronaći inverz matrice
dinamičke krutosti. Inverz matrice dinamičke krutosti predstavlja matrica
frekvencijskih funkcija odziva, a označavamo ju s $\mathbf{H}$. Jednadžba pod 
\eqref{eq:mat_din_krutost} može poprimiti slijedeći oblik:
\begin{equation}
    \left(\ppsi^N\right)^T[\kk-\omega^2\mm]\ppsi^N
    =
    \left(\ppsi^N\right)^T\mathbf{Z}\;\ppsi^T
\end{equation}
Odnosno
\begin{equation}
     \left(\ppsi^N\right)^T[\kk-\omega^2\mm]\ppsi^N
    =
    \left(\ppsi^N\right)^T\mathbf{H}^{-1}\;\ppsi^T
\end{equation}

Iz \eqref{eq:relacija_normirani_matricno} i \eqref{eq:normirana_krutost} slijedi:
\begin{equation}
    [\omega_n^2 - \omega^2] = \left(\ppsi^N\right)^T\mathbf{H}^{-1}\;\ppsi^T
\end{equation}

Nakon sređivanja, dobijemo:
\begin{equation}
    \mathbf{H}=\left(\ppsi^N\right)^T[\omega_n^2 - \omega^2]^{-1}\;\ppsi^T
\end{equation}

Član $H_{j,k}$ matrice $\mathbf{H}$ možemo dobiti preko slijedeće jednadžbe:
\begin{equation}\label{eq:clan_frf_matrice}
    H_{j,k}=\frac{\psi^N_{j,1}\psi^N_{k,1}}{\omega_1^2-\omega^2}
            +
            \frac{\psi^N_{j,2}\psi^N_{k,2}}{\omega_2^2-\omega^2}
            +
            \frac{\psi^N_{j,3}\psi^N_{k,3}}{\omega_3^2-\omega^2}
            +
            \cdots
            +
            \frac{\psi^N_{j,n}\psi^N_{k,n}}{\omega_n^2-\omega^2}
\end{equation}

Jednadžbu \eqref{eq:clan_frf_matrice} možemo zapisati kao skalarni umnožak vektora:
\begin{equation}\label{eq:clan_frf_matrica_vektorski}
    H_{j,k}
    =
    \begin{Bmatrix}
        \psi_{j,1}\psi_{k,1} &
        \psi_{j,2}\psi_{k,2} &
        \psi_{j,3}\psi_{k,3} &
        \cdots
        \psi_{j,n}\psi_{k,n}
    \end{Bmatrix}
    \begin{Bmatrix}
        \ffrac{1}{\omega_1^2-\omega^2}\\[6pt]
        \ffrac{1}{\omega_2^2-\omega^2}\\[6pt]
        \ffrac{1}{\omega_3^2-\omega^2}\\[6pt]
        \vdots\\[6pt]
        \ffrac{1}{\omega_n^2-\omega^2}
    \end{Bmatrix}
\end{equation}

Za zadani sustav, matrica modova normiranih s obzirom na masu $\ppsi^N$ glasi:
\[
    \ppsi^N
    =
    \begin{Bmatrix}
        \ffrac{\sqrt{6m}}{6m} & -\ffrac{\sqrt{3m}}{3m}\\[12pt]
        \ffrac{\sqrt{6m}}{6m} & \ffrac{\sqrt{3m}}{3m}
    \end{Bmatrix}
\]

Izračun matrice frekvencijske funkcije odziva $H_{1,1}$ prikazan je u nastavku:
\[
\begin{aligned}
    H_{1,1} &=
    \frac{\psi^N_{1,1}}{\omega_1^2-\omega^2}+\frac{\psi^N_{1,1}}{\omega_2^2-\omega^2}
    =
    \frac{\ffrac{\sqrt{6m}}{6m}\ffrac{\sqrt{6m}}{6m}}{\omega_1^2-\omega^2}
    +
    \frac{\ffrac{\sqrt{3m}}{3m}\ffrac{\sqrt{3m}}{3m}}{\omega_2^2-\omega^2}\\
    %
    H_{1,1} &=
    \frac{1}{3m} 
        \left(
            \frac{\ffrac{1}{2}}{(\omega_1^2-\omega^2}
            +
            \frac{1}{(\omega_2^2-\omega^2)}
        \right)
    =
    \frac{1}{3m}
        \left(
            \frac{\ffrac{1}{2}(\omega_2^2-\omega^2)+(\omega_1^2-\omega^2)}
                {\omega_1^2\omega_2^2(1-(\omega/\omega_1)^2)(1-(\omega/\omega_2)^2}
        \right)\\
\end{aligned}
\]

U brojniku, $\omega_2$ izrazimo kao $\omega_1$ pomoću relacije $\omega_2=\sqrt{2k/m}$
i $\omega_1=\sqrt{k/2m}$.
\[
    \begin{aligned}
        H_{1,1}=
        \frac{1}{3m}
            \left(
                \frac{3-\ffrac{3}{2}(\omega/\omega_1)^2}
                    {\omega_2^2(1-(\omega/\omega_1)^2)(1-(\omega/\omega_2)^2)}
            \right)
    \end{aligned}
\]
Nakon sređivanja:
\begin{equation}\label{eq:frf_11}
    H_{1,1}=\frac{1-\ffrac{1}{2}(\omega/\omega_1)^2}{2k(1-(\omega_1/\omega)^2)(1-(\omega_2/\omega)^2)}
\end{equation}

Analogno tome, dobiju se ostali članovi matrice te glase:
\begin{align}
    H_{1,2} &= \frac{1}{2k(1-(\omega_1/\omega)^2)(1-(\omega_2/\omega)^2)}\label{eq:frf_12}\\
    H_{2,1} &= \frac{1}{2k(1-(\omega_1/\omega)^2)(1-(\omega_2/\omega)^2)}\label{eq:frf_21}\\
    H_{2,2} &= \frac{1-2(\omega/\omega_2)^2}{2k(1-(\omega_1/\omega)^2)(1-(\omega_2/\omega)^2)}\label{eq:frf_22}\\
\end{align}

Matrica $\mathbf{H}$ glasi:
\begin{equation}\label{eq:matrica_frf}
    \mathbf{H}=\frac{1}{2k(1-(\omega/\omega_1)^2(1-(\omega/\omega_2)^2)}
    \begin{bmatrix}
        1-\ffrac{1}{2}\left(\ffrac{\omega}{\omega_1}\right)^2 & 1 \\
        1 & 1-2\left(\ffrac{\omega}{\omega_2}\right)^2
    \end{bmatrix}
\end{equation}

Uvrštavanjem \eqref{eq:matrica_frf} u \eqref{eq:jdn_gibanja_part_1} dobijemo:
\begin{equation}
    \mathbf{H}^{-1}\vtor{U}{}=\vtor{p}{}
\end{equation}

Množenjem prethodnog izraza s $\mathbf{H}$:
\begin{equation}
    \vtor{U}{} = \mathbf{H}\vtor{p}{}
\end{equation}

Raspisivanjem prethodne jednadžbe:
\begin{equation}
    \begin{Bmatrix}
        U_1\\
        U_2
    \end{Bmatrix}
    =
    \frac{1}{2k\left(1-\left(\ffrac{\omega_1}{\omega}\right)^2\right)\left(1-\left(\ffrac{\omega_2}{\omega}\right)^2\right)}
    %
    \begin{bmatrix}
        1-\ffrac{1}{2}\left(\ffrac{\omega}{\omega_1}\right)^2 & 1 \\
        1 & 1-2\left(\ffrac{\omega}{\omega_2}\right)^2
    \end{bmatrix}
    % 
    \begin{Bmatrix}
        p_0\\
        0
    \end{Bmatrix}
\end{equation}

Te konačno, vektor $\vtor{U}{}$ glasi:
\begin{equation}\label{eq:vektor_u_konacno}
    \begin{Bmatrix}
        U_1\\
        U_2
    \end{Bmatrix}
    =
    \frac{1}{2k\left(1-\left(\ffrac{\omega_1}{\omega}\right)^2\right)\left(1-\left(\ffrac{\omega_2}{\omega}\right)^2\right)}
    \begin{Bmatrix}
        p_0\left(1-\ffrac{1}{2}\left(\ffrac{\omega}{\omega_1}\right)^2\right)\\
        p_0 
    \end{Bmatrix}
\end{equation}

Dijeljenjem vektora $\vtor{U}{}$ s $p_0/2k$ dobijemo slijedeće:
\begin{equation}
    \begin{Bmatrix}
        U_1\\
        U_2
    \end{Bmatrix}
    =
    \frac{1}{\left(1-\left(\ffrac{\omega_1}{\omega}\right)^2\right)\left(1-\left(\ffrac{\omega_2}{\omega}\right)^2\right)}
    \begin{Bmatrix}
        1-\ffrac{1}{2}\left(\ffrac{\omega}{\omega_1}\right)^2\\
        1 
    \end{Bmatrix}
\end{equation}

%Komponente vektora zapisane posebno:
%\begin{align}
%    U_1&=\frac{1-\ffrac{1}{2}\left(\ffrac{\omega}{\omega_1}\right)^2}
%        {\left(1-\left(\ffrac{\omega_1}{\omega}\right)^2\right)\left(1-\left(\ffrac{\omega_2}{\omega}\right)^2\right)}\\
%    U_2&=\frac{1}{\left(1-\left(\ffrac{\omega_1}{\omega}\right)^2\right)\left(1-\left(\ffrac{\omega_2}{\omega}\right)^2\right)}
%\end{align}

Komponente vektora $U$ predstavljaju funkcije ovisnosti dinamičkog faktora o 
frekvencijskim omjerima $\omega/\omega_1$ i $\omega/\omega_2$. Njihovi grafovi
prikazani su u nastavku.

\textbf{UBACI GRAFOVE}

Prisilni dio odziva glasi:
\begin{equation}
    \begin{split}
        \vtor{u(t)}{} = \vtor{U}{}\; sin(\omega t)
        %\vtor{u(t)}{}=
        =
        \frac{1}{\left(1-\left(\ffrac{\omega_1}{\omega}\right)^2\right)\left(1-\left(\ffrac{\omega_2}{\omega}\right)^2\right)}
    \begin{Bmatrix}
        1-\ffrac{1}{2}\left(\ffrac{\omega}{\omega_1}\right)^2\\
        1 
    \end{Bmatrix}
    \sin(\omega t)
    \end{split}
\end{equation}
