\section{Ortogonalnost vlastitih vektora}
Kao što je već pokazano, oblike osciliranja definiraju vlastiti vektori. Dva vektora su
međusobno ortogonalna (okomita) ukoliko je njihov skalarni produkt jednak nuli.
Razmotrimo li $r$-ti i $n$-ti vlastiti vektor sustava, dobijemo slijedeći sustav jednadžbi 
(iz \eqref{eq:sustav_diferencijalnih_matricni_kratko}):
\begin{equation}\label{eq:pocetni_sustav_ortogonalnost}
    \begin{dcases}
        (\kk-\omega_r^2\mm)\vtor{\psi}{}_r=\vtor{0}{}\\
        (\kk-\omega_n^2\mm)\vtor{\psi}{}_n=\vtor{0}{}
    \end{dcases}
\end{equation}

Donju jednadžbu pomnožimo s $\vtor{\psi}{}_r^T$. U gornjoj jednadžbi prvo
transponiramo $\vtor{\psi}{}_r$ te ju pomnožimo s $\vtor{\psi}{}_n$. Sustav jednadžbi
\eqref{eq:pocetni_sustav_ortogonalnost} postaje:
\begin{equation}\label{eq:konacni_sustav_ortogonalnost}
    \begin{dcases}
        \vtor{\psi}{}_r^T(\kk-\omega_r^2\mm)\vtor{\psi}{}_n=0\\
        \vtor{\psi}{}_r^T(\kk-\omega_n^2\mm)\vtor{\psi}{}_n=0
    \end{dcases}
\end{equation}

Oduzimanjem gornje i donje jednadžbe dobijemo:
\begin{equation}
    (\omega_n^2-\omega_r^2)\vtor{\psi}{}_r^T\mm\vtor{\psi}{}_n=0
\end{equation}

Za $\omega_n\neq\omega_r$ vrijedi:
\begin{equation}\label{eq:ortogonalnost_masa}
    \vtor{\psi}{}_r^T\mm\vtor{\psi}{}_n=0
\end{equation}

Uvrštavanjem $\vtor{\psi}{}_r^T\mm\vtor{\psi}{}_n=0$ u bilo koju od jednadžbi iz
\eqref{eq:konacni_sustav_ortogonalnost} dobijemo:
\begin{equation}\label{eq:ortogonalnost_krutost}
    \vtor{\psi}{}_r^T\kk\vtor{\psi}{}_n=0
\end{equation}

Jednadžbe pod \eqref{eq:ortogonalnost_masa} i \eqref{eq:ortogonalnost_krutost}
govore da su vlastiti vektori m-ortogonalni i k-ortogonalni. Odnosno, kažemo da su
vlastiti vektori međusobno ortogonalni s obzirom na matricu mase ili matricu
krutosti. 
\par

Poslijedica ortogonalnosti su slijedeće dijagonalne pravokutne matrice:
\begin{alignat}{2}
    &\text{Modalna krutost}\quad & \mathbf{K}&=\ppsi^T\kk\ppsi\label{eq:modalna_krutost_matrica}\\
    &\text{Modalna masa}\quad &\mathbf{M}&=\ppsi^T\mm\ppsi\label{eq:modalna_masa_matrica}
\end{alignat}

Članovi na dijagonali računaju se prema slijedećim formulama:
\begin{alignat}{2}
    &\text{Za modalnu krutost}\quad & K_{n,n}&=\vtor{\psi}{}_n^Tk\vtor{\psi}{}_n\label{eq:modalna_krutost}\\
    &\text{Za modalnu masu}\quad &M_{n,n}&=\vtor{\psi}{}_n^Tm\vtor{\psi}{}_n\label{eq:modalna_masa}
\end{alignat}

Između elemenata matrica vrijedi slijedeći odnos:
\begin{equation}
    \omega_n^2=\frac{K_n}{M_n}
\end{equation}
U matričnoj formi:
\begin{equation}
    \oomega^2=\mathbf{K}\mathbf{M}^{-1}
\end{equation}

