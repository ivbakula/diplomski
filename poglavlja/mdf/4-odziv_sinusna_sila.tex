\section{Odziv sustava s više stupnjeva slobode na pobudu sinusnom
silom}\label{mdof_prisilne}
Kao što je pokazano u poglavlju \ref{slobodne_oscilacije}, jednadžba gibanja sustava
s $N$ stupnjeva slobode zadana je kao sustav od $N$ diferencijalnih jednadži drugog
reda. U slučaju pobude harmonijskom silom, navedeni sustav će se sastojati od $N$
nehomogenih diferencijalnih jednadžbi drugog reda, koje će biti povezane preko
matrice krutosti i/ili matrice mase. 
\par
Općenito, rješenje jedne proizvoljne nehomogene diferencijalne jednadžbe drugog reda
oblika $\alpha\ddot{y}+\beta\dot{y}+\gamma y= f(t)$ jest suma komplementarnog
rješenja $y_c$ i partikularnog rješenja $Y_p$.
\begin{equation}
    y(t)=y_c(t)+U_p(t)
\end{equation}

Komplementarno rješenje dobijemo izjednačavanjem diferencijalne jednadžbe s nulom
odnosno:
\begin{equation}\label{eq:opce_komplementarno_rjesenje}
    \alpha\ddot{y}+\beta\dot{y}+\gamma y=0
\end{equation}

Primjetimo da je komplementarno rješenje (rješenje jednadžbe \eqref{eq:opce_komplementarno_rjesenje}) 
zapravo rješenje homogene diferencijalne jednadžbe, a jednako je i za slobodne 
oscilacije i za prisilne oscilacije. Kod prisilnih oscilacija, komplementarno 
rješenje predstavlja prolazni dio odziva. Partikularno riješenje možemo pronaći 
koristeći se metodom neodređenih koeficijenata, a predstavlja prolazni dio odziva.
\par

Analogno tome, komplementarno rješenje sustava diferencijalnih jednadžbi dato je u
\eqref{eq:opce_rjesenje_sustava} te predstavlja prolazni dio odziva, pa u slučaju
prisilnih oscilacija preostaje nam odrediti još partikularno rješenje koje
predstavlja prisilni dio odziva. 
\par

Zadana je jednadžba gibanja sustava s dva stupnja slobode prikazanog na slijedećoj
slici

\textbf{UBACI SLIKU SUSTAVA}

\begin{equation}\label{eq:jednadzba_gibanja_matricno}
    \begin{bmatrix}
        m_1 & 0 \\
        0   & m_2
    \end{bmatrix}
    \begin{Bmatrix}
        \ddot{u}_1\\
        \ddot{u}_2
    \end{Bmatrix}
    +
    \begin{bmatrix}
        k_1+k_2 & -k_2\\
        -k_2 & k_2
    \end{bmatrix}
    \begin{Bmatrix}
        u_1\\
        u_2
    \end{Bmatrix}
    =
    \begin{Bmatrix}
        p_0\\
        0 
    \end{Bmatrix}
    \sin(\omega t)
\end{equation}

Odnosno:
\begin{equation}\label{eq:jednadzba_gibanja_matricno_kratko}
    \mm\vtor{u}{:}+k\vtor{u}{}=\vtor{p_n}{}\sin(\omega t)
\end{equation}

Gdje je $\vtor{p_n}{}$ vektor amplituda harmonijskih sila. Odziv sustava biti će
harmonijski, jednake frekvencije, pa partikularno rješenje možemo pretpostaviti:
\begin{equation}\label{eq:partikularno_pretpostavka}
    \vtor{u_p(t)}{}=\vtor{U_n}{}\sin(\omega t)
\end{equation}
Gdje je $\vtor{U_n}{}$ vektor koeficijenata.
Druga derivacija \eqref{eq:partikularno_pretpostavka} glasi:
\begin{equation}\label{eq:partikularno_pretpostavka_dd}
    \{\ddot{u}(t)\}=-\omega^2\vtor{U_n}{}\sin(\omega t)
\end{equation}

Uvrštavanjem \eqref{eq:partikularno_pretpostavka} i \eqref{eq:partikularno_pretpostavka_dd}
u \eqref{eq:jednadzba_gibanja_matricno_kratko} dobijemo:
\begin{equation}\label{eq:neodredjeni_koeficijenti_1}
    -\omega^2\vtor{U_n}{}\mm\sin(\omega t) 
    +
    \vtor{U_n}{}\kk\sin(\omega t)
    =
    \vtor{p}{}\sin(\omega t)
\end{equation}

Nakon sređivanja, jednadžba \eqref{eq:neodredjeni_koeficijenti_1} poprima oblik:
\begin{equation}\label{eq:neodredjeni_koeficijenti_2}
    [\kk-\omega^2\mm]\vtor{U_n}{}=\vtor{p}{}
\end{equation}

%Matrica $[\kk-\omega^2\mm]$ predstavlja matricu dinamičke krutosti sustava s više stupnjeva
%slobode, a označava se kao $[Z(\omega)]$. Jednadžbu pod \eqref{eq:neodredjeni_koeficijenti_2}
%možemo zapisati kao:
%\begin{equation}\label{eq:neodredjeni_koeficijenti_z}
%    [Z(\omega)]\vtor{U_n}{}=\vtor{p_n}{}
%\end{equation}
%
%Konačno, vektor koeficijenata dobijemo množenjem izraza \eqref{eq:neodredjeni_koeficijenti_z}
%inversom matrice dinamičke krutosti, te dobijemo:
%\begin{equation}
%    \vtor{U_n}{}=\vtor{p_n}{}\,[Z(\omega)]^{-1}
%\end{equation}

Množenjem jednadžbe \eqref{eq:neodredjeni_koeficijenti_2} s $[\kk-\omega^2\mm]^{-1}$
dobijemo:
\begin{equation}
    \begin{split}
        \vtor{U_n}{}=[\kk-\omega^2\mm]^{-1}\vtor{p_n}{} \notag\\
        \vtor{U_n}{}=\frac{1}{det[\kk-\omega^2\mm]}adj[\kk-\omega^2\mm]\vtor{p_n}{}\label{eqneodredjeni_koeficijenti_3}
    \end{split}
\end{equation}

Odnosno u matričnom obliku:
\begin{equation}
    \begin{Bmatrix}
        U_1\\
        U_2
    \end{Bmatrix}
    =
    \frac{1}{det[\kk-\omega^2\mm]}
        \begin{bmatrix}
            k_2-m_2\omega^2 & k_2\\
            k_2 & k_1+k_2-m_1\omega^2
        \end{bmatrix}
        \begin{Bmatrix}
            p_0\\
            0
        \end{Bmatrix}
\end{equation}

Komponente vektora $\vtor{U_n}{}$ glase:
\begin{align}
    U_1=\frac{p_0(k_2-m_2\omega^2)}{m_1m_2(\omega^2-\omega_1^2)(\omega^2-\omega_2^2)} \label{eq:vtor_1_opce}\\
    U_2=\frac{p_0k_2}{m_1m_2(\omega^2-\omega_1^2)(\omega^2-\omega_2^2)}\label{eq:vtor_2_opce}
\end{align}

Za $m_1=2m$, $m_2=m$, $k_1=2k$ i $k_2=k$ vektori glase:
\begin{align}
    U_1=\frac{p_0(k-m\omega^2)}{2m^2(\omega^2-\omega_1^2)(\omega^2-\omega_2^2)}\label{eq:vtor_1}\\
    U_2=\frac{p_0k}{2m^2(\omega^2-\omega_1^2)(\omega^2-\omega_2^2)}\label{eq:vtor_2}
\end{align}

Uz $\omega_1=\sqrt{k/2m}$ i $\omega_2=\sqrt{2k/m}$ te dijeljenjem \eqref{eq:vtor_1} i \eqref{eq:vtor_2}
dobijemo vektor dinamičkog koeficijenta pomaka (bez dimenzija), koji ovisi o
omjerima frekvencija $\omega/\omega_1$ te $\omega/\omega_2$. Vektor je prikazan u nastavku
\begin{equation}
    \frac{2k}{p_0}\vtor{U}{}
    =
    \begin{Bmatrix}
        \ffrac{1-0.5(\omega/\omega_1)^2}
              {[1-(\omega/\omega_1)^2][1-(\omega/\omega_2)^2]}\\[12pt]
        \ffrac{1}
              {[1-(\omega/\omega_1)^2][1-(\omega/\omega_2)^2]}
    \end{Bmatrix}
\end{equation}

Prisilni dio odziva glasi:
\begin{equation}
    \vtor{u(t)}{} = \vtor{U}{}\sin(\omega t) = 
    \begin{Bmatrix}
        \ffrac{1-0.5(\omega/\omega_1)^2}
              {[1-(\omega/\omega_1)^2][1-(\omega/\omega_2)^2]}\\[12pt]
        \ffrac{1}
              {[1-(\omega/\omega_1)^2][1-(\omega/\omega_2)^2]}
    \end{Bmatrix}
    \sin(\omega t)
\end{equation}

Komponente vektora $\vtor{U}{}$ možemo iscrtati kao graf funkcije dinamičkog faktora
$U_1/p_0/2k$ i $U_2/p_0/2k$ u ovisnosti o frekvencijskom omjeru $\omega/\omega_1$.

\section{Interesantan postupak za rješavanje 2dof}
Zadana je jednadžba gibanja prisilnog titranja sustava s dva stupnja slobode.
\begin{equation}\label{eq:jdn_gibanja_osnovno}
    \mm\vtor{u}{:}+\kk\vtor{u}{}=\vtor{p}{}\sin(\omega t)
\end{equation}

Odziv sustava će biti sinusni, iste frekvencije kao i pobuda, pa rješenje
pretpostavljamo u slijedećem obliku:
\begin{equation}\label{eq:pretpostavljeno}
    \vtor{U_p}{}=\vtor{U}{}\sin(\omega t)
\end{equation}

Druga derivacija pretpostavljenog rješenja glasi:
\begin{equation}\label{eq:pretpostavljeno_dd}
    \vtor{U_p}{}=-\omega^2\vtor{U}{}\sin(\omega t)
\end{equation}

Uvrštavanjem \eqref{eq:pretpostavljeno} i \eqref{eq:pretpostavljeno_dd} u
\eqref{eq:jdn_gibanja_osnovno} dobijemo:
\begin{equation}
    -\omega^2\vtor{U}{}\mm\sin(\omega t) + \mm\vtor{U}{}\sin(\omega t) = \vtor{p}{}\sin(\omega t)
\end{equation}

Te nakon sređivanja:
\begin{equation}\label{eq:jdn_gibanja_part_1}
    [\kk-\omega^2\mm]\vtor{U}{}=\vtor{p}{}
\end{equation}

Matrica $[\kk-\omega^2\mm]$ predstavlja matricu dinamičke krutosti.
\begin{equation}\label{eq:mat_din_krutost}
    [\kk-\omega^2\mm]=\mathbf{Z}
\end{equation}

Da bismo rješili \eqref{eq:jdn_gibanja_part_1} potrebno je pronaći invers matrice
dinamičke krutosti. Invers matrice dinamičke krutosti predstavlja matrica
frekvencijskih funkcija odziva, a označavamo ju s $\mathbf{H}$. Jednadžba pod 
\eqref{eq:mat_din_krutost} može poprimiti slijedeći oblik:
\begin{equation}
    \left(\ppsi^N\right)^T[\kk-\omega^2\mm]\ppsi^N
    =
    \left(\ppsi^N\right)^T\mathbf{Z}\;\ppsi^T
\end{equation}
Odnosno
\begin{equation}
     \left(\ppsi^N\right)^T[\kk-\omega^2\mm]\ppsi^N
    =
    \left(\ppsi^N\right)^T\mathbf{H}^{-1}\;\ppsi^T
\end{equation}

Iz \eqref{eq:relacija_normirani_matricno} i \eqref{eq:normirana_krutost} slijedi:
\begin{equation}
    [\omega_n^2 - \omega^2] = \left(\ppsi^N\right)^T\mathbf{H}^{-1}\;\ppsi^T
\end{equation}

Nakon sređivanja, dobijemo:
\begin{equation}
    \mathbf{H}=\left(\ppsi^N\right)^T[\omega_n^2 - \omega^2]^{-1}\;\ppsi^T
\end{equation}

Član $H_{j,k}$ matrice $\mathbf{H}$ možemo dobiti preko slijedeće jednadžbe:
\begin{equation}\label{eq:clan_frf_matrice}
    H_{j,k}=\frac{\psi^N_{j,1}\psi^N_{k,1}}{\omega_1^2-\omega^2}
            +
            \frac{\psi^N_{j,2}\psi^N_{k,2}}{\omega_2^2-\omega^2}
            +
            \frac{\psi^N_{j,3}\psi^N_{k,3}}{\omega_3^2-\omega^2}
            +
            \cdots
            +
            \frac{\psi^N_{j,n}\psi^N_{k,n}}{\omega_n^2-\omega^2}
\end{equation}

Jednadžbu \eqref{eq:clan_frf_matrice} možemo zapisati kao skalarni umnožak vektora:
\begin{equation}\label{eq:clan_frf_matrica_vektorski}
    H_{j,k}
    =
    \begin{Bmatrix}
        \psi_{j,1}\psi_{k,1} &
        \psi_{j,2}\psi_{k,2} &
        \psi_{j,3}\psi_{k,3} &
        \cdots
        \psi_{j,n}\psi_{k,n}
    \end{Bmatrix}
    \begin{Bmatrix}
        \ffrac{1}{\omega_1^2-\omega^2}\\[6pt]
        \ffrac{1}{\omega_2^2-\omega^2}\\[6pt]
        \ffrac{1}{\omega_3^2-\omega^2}\\[6pt]
        \vdots\\[6pt]
        \ffrac{1}{\omega_n^2-\omega^2}
    \end{Bmatrix}
\end{equation}

Za zadani sustav, matrica modova normiranih s obzirom na masu $\ppsi^N$ glasi:
\[
    \ppsi^N
    =
    \begin{Bmatrix}
        \ffrac{\sqrt{6m}}{6m} & -\ffrac{\sqrt{3m}}{3m}\\[12pt]
        \ffrac{\sqrt{6m}}{6m} & \ffrac{\sqrt{3m}}{3m}
    \end{Bmatrix}
\]

Izračun matrice frekvencijske funkcije odziva $H_{1,1}$ prikazan je u nastavku:
\[
\begin{aligned}
    H_{1,1} &=
    \frac{\psi^N_{1,1}}{\omega_1^2-\omega^2}+\frac{\psi^N_{1,1}}{\omega_2^2-\omega^2}
    =
    \frac{\ffrac{\sqrt{6m}}{6m}\ffrac{\sqrt{6m}}{6m}}{\omega_1^2-\omega^2}
    +
    \frac{\ffrac{\sqrt{3m}}{3m}\ffrac{\sqrt{3m}}{3m}}{\omega_2^2-\omega^2}\\
    %
    H_{1,1} &=
    \frac{1}{3m} 
        \left(
            \frac{\ffrac{1}{2}}{(\omega_1^2-\omega^2}
            +
            \frac{1}{(\omega_2^2-\omega^2)}
        \right)
    =
    \frac{1}{3m}
        \left(
            \frac{\ffrac{1}{2}(\omega_2^2-\omega^2)+(\omega_1^2-\omega^2)}
                {\omega_1^2\omega_2^2(1-(\omega/\omega_1)^2)(1-(\omega/\omega_2)^2}
        \right)\\
\end{aligned}
\]

U brojniku, $\omega_2$ izrazimo kao $\omega_1$ pomoću relacije $\omega_2=\sqrt{2k/m}$
i $\omega_1=\sqrt{k/2m}$.
\[
    \begin{aligned}
        H_{1,1}=
        \frac{1}{3m}
            \left(
                \frac{3-\ffrac{3}{2}(\omega/\omega_1)^2}
                    {\omega_2^2(1-(\omega/\omega_1)^2)(1-(\omega/\omega_2)^2)}
            \right)
    \end{aligned}
\]
Nakon sređivanja:
\begin{equation}\label{eq:frf_11}
    H_{1,1}=\frac{1-\ffrac{1}{2}(\omega/\omega_1)^2}{2k(1-(\omega_1/\omega)^2)(1-(\omega_2/\omega)^2)}
\end{equation}

Analogno tome, dobiju se ostali članovi matrice te glase:
\begin{align}
    H_{1,2} &= \frac{1}{2k(1-(\omega_1/\omega)^2)(1-(\omega_2/\omega)^2)}\label{eq:frf_12}\\
    H_{2,1} &= \frac{1}{2k(1-(\omega_1/\omega)^2)(1-(\omega_2/\omega)^2)}\label{eq:frf_21}\\
    H_{2,2} &= \frac{1-2(\omega/\omega_2)^2}{2k(1-(\omega_1/\omega)^2)(1-(\omega_2/\omega)^2)}\label{eq:frf_22}\\
\end{align}

Matrica $\mathbf{H}$ glasi:
\begin{equation}\label{eq:matrica_frf}
    \mathbf{H}=\frac{1}{2k(1-(\omega/\omega_1)^2(1-(\omega/\omega_2)^2)}
    \begin{bmatrix}
        1-\ffrac{1}{2}\left(\ffrac{\omega}{\omega_1}\right)^2 & 1 \\
        1 & 1-2\left(\ffrac{\omega}{\omega_2}\right)^2
    \end{bmatrix}
\end{equation}

Uvrštavanjem \eqref{eq:matrica_frf} u \eqref{eq:jdn_gibanja_part_1} dobijemo:
\begin{equation}
    \mathbf{H}^{-1}\vtor{U}{}=\vtor{p}{}
\end{equation}

Množenjem prethodnog izraza s $\mathbf{H}$:
\begin{equation}
    \vtor{U}{} = \mathbf{H}\vtor{p}{}
\end{equation}

Raspisivanjem prethodne jednadžbe:
\begin{equation}
    \begin{Bmatrix}
        U_1\\
        U_2
    \end{Bmatrix}
    =
    \frac{1}{2k\left(1-\left(\ffrac{\omega_1}{\omega}\right)^2\right)\left(1-\left(\ffrac{\omega_2}{\omega}\right)^2\right)}
    %
    \begin{bmatrix}
        1-\ffrac{1}{2}\left(\ffrac{\omega}{\omega_1}\right)^2 & 1 \\
        1 & 1-2\left(\ffrac{\omega}{\omega_2}\right)^2
    \end{bmatrix}
    % 
    \begin{Bmatrix}
        p_0\\
        0
    \end{Bmatrix}
\end{equation}

Te konačno, vektor $\vtor{U}{}$ glasi:
\begin{equation}\label{eq:vektor_u_konacno}
    \begin{Bmatrix}
        U_1\\
        U_2
    \end{Bmatrix}
    =
    \frac{1}{2k\left(1-\left(\ffrac{\omega_1}{\omega}\right)^2\right)\left(1-\left(\ffrac{\omega_2}{\omega}\right)^2\right)}
    \begin{Bmatrix}
        p_0\left(1-\ffrac{1}{2}\left(\ffrac{\omega}{\omega_1}\right)^2\right)\\
        p_0 
    \end{Bmatrix}
\end{equation}

Dijeljenjem vektora $\vtor{U}{}$ s $p_0/2k$ dobijemo slijedeće:
\begin{equation}
    \begin{Bmatrix}
        U_1\\
        U_2
    \end{Bmatrix}
    =
    \frac{1}{\left(1-\left(\ffrac{\omega_1}{\omega}\right)^2\right)\left(1-\left(\ffrac{\omega_2}{\omega}\right)^2\right)}
    \begin{Bmatrix}
        1-\ffrac{1}{2}\left(\ffrac{\omega}{\omega_1}\right)^2\\
        1 
    \end{Bmatrix}
\end{equation}

%Komponente vektora zapisane posebno:
%\begin{align}
%    U_1&=\frac{1-\ffrac{1}{2}\left(\ffrac{\omega}{\omega_1}\right)^2}
%        {\left(1-\left(\ffrac{\omega_1}{\omega}\right)^2\right)\left(1-\left(\ffrac{\omega_2}{\omega}\right)^2\right)}\\
%    U_2&=\frac{1}{\left(1-\left(\ffrac{\omega_1}{\omega}\right)^2\right)\left(1-\left(\ffrac{\omega_2}{\omega}\right)^2\right)}
%\end{align}

Komponente vektora $U$ predstavljaju funkcije ovisnosti dinamičkog faktora o 
frekvencijskim omjerima $\omega/\omega_1$ i $\omega/\omega_2$. Njihovi grafovi
prikazani su u nastavku.

\textbf{UBACI GRAFOVE}

Prisilni dio odziva glasi:
\begin{equation}
    \begin{split}
        \vtor{u(t)}{} = \vtor{U}{}\; sin(\omega t)
        %\vtor{u(t)}{}=
        =
        \frac{1}{\left(1-\left(\ffrac{\omega_1}{\omega}\right)^2\right)\left(1-\left(\ffrac{\omega_2}{\omega}\right)^2\right)}
    \begin{Bmatrix}
        1-\ffrac{1}{2}\left(\ffrac{\omega}{\omega_1}\right)^2\\
        1 
    \end{Bmatrix}
    \sin(\omega t)
    \end{split}
\end{equation}

