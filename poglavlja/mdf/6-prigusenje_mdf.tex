\section{Prigušenje u sustavu s više stupnjeva slobode}
Model prigušenog sustava s više stupnjeva slobode prikazan je na slijedećoj slici.
\par
\begin{figure}[H]
    \begin{subfigure}[b]{0.5\textwidth}
        \centering
        \begin{tikzpicture}
%štapovi
    %prvi (donji) okvir
	\draw[thick] (0,0) -- (0,2.2)
                node[pos=0.5,left]{\large{$k_1$}};
	\draw[thick] (0,2.2) -- (4,2.2);
	\draw[thick] (4,0) -- (4,2.2);

    %drugi (gornji) okvir
        \draw[thick] (0,2.2) -- (0, 4.4)
                node[pos=0.5,left]{\large{$k_2$}};
        \draw[thick] (0, 4.4) -- (4, 4.4);
        \draw[thick] (4, 2.2) -- (4, 4.4);
	
%mase
        %donja
	\filldraw[color=black, fill=gray] (2,2.2) circle (0.42);
	\node[draw=none, fill=none] at(2,2.2) 
        {$m_1$};

        %gornja
        \filldraw[color=black, fill=gray] (2,4.4) circle (0.42);
        \node[draw=none, fill=none] at (2,4.4)
        {$m_2$};

        %dolje
        \prigusivac{0}{0}{4}{2.2}
        \node[draw=none, fill=none] at (1.65, 1.4) {\large{$c_1$}};
        %gore
        \prigusivac{0}{2.2}{4}{4.4} 
        \node[draw=none, fill=none] at (1.65, 3.6) {\large{$c_2$}};


%podloga
	\draw[white, pattern=north east lines, pattern color=black] (-0.5, 0) 
	rectangle (0.5, -0.5);
	\draw[thick] (-0.5, 0) -- (0.5, 0);

	\draw[white, pattern=north east lines, pattern color=black] (3.5, 0)
	rectangle (4.5, -0.5);
	\draw[thick] (3.5, 0) -- (4.5, 0);
\end{tikzpicture}

        \caption{}
        \label{fig:priguseni_sustav_okvir-2dof}
    \end{subfigure}
    \hfill
    \begin{subfigure}[b]{0.5\textwidth}
        \centering
        \begin{tikzpicture}
	%podloga
	\draw[white, pattern=north east lines, pattern color=black] (0, 0)
	rectangle (-0.5, 2.2);
	\draw[thick] (0,0) -- (0, 2.2);

	\draw[white, pattern=north east lines, pattern color=black] (0, 0) 
	rectangle (8, -0.5);
	\draw[thick] (0,0) -- (8,0);
	
	\draw[white, pattern=north east lines, pattern color=black] (0, 0)
	rectangle (-0.5, -0.5);

	%uteg
	\filldraw[fill=gray] (1.5, 2) rectangle (3.5, 0.25);
        \filldraw[fill=gray] (5, 2) rectangle (7, 0.25);

	%kotaci
	\filldraw[fill=gray] (2, 0.125) circle (0.125);
	\filldraw[fill=gray] (3, 0.125) circle (0.125);

        \filldraw[fill=gray] (5.5, 0.125) circle (0.125);
        \filldraw[fill=gray] (6.5, 0.125) circle (0.125);

	%opruga
	\draw[thick, decoration={aspect=0.1, segment length=2mm,amplitude=3mm,coil},decorate] (0,1.65) -- (1.5, 1.65);
        \draw[thick, decoration={aspect=0.3, segment length=2mm,amplitude=3mm,coil},decorate] (3.5, 1.65) -- (5, 1.65);

        %prigusivac 1
	\draw[thick] (0, 0.65) -- (0.685, 0.65);
	\draw[thick] (0.84, 0.65) -- (1.5, 0.65);

	\draw[thick] (0.685, 0.35) -- (0.685, 0.95);
	\draw[thick] (0.685, 0.95) -- (0.9, 0.95);
	\draw[thick] (0.685, 0.35) -- (0.9, 0.35);

	\draw[thick] (0.84, 0.5) -- (0.84, 0.8);

        %prigusivac 2 
	\draw[thick] (3.5, 0.65) -- (4.185, 0.65);
	\draw[thick] (4.34, 0.65) -- (5, 0.65);

	\draw[thick] (4.185, 0.35) -- (4.185, 0.95);
	\draw[thick] (4.185, 0.95) -- (4.4, 0.95);
	\draw[thick] (4.185, 0.35) -- (4.4, 0.35);

	\draw[thick] (4.34, 0.5) -- (4.34, 0.8);

        \node[draw=none, fill=none] at (0.75, 2.2) {\large{$k_1$}}; 
        \node[draw=none, fill=none] at (2.5,   1) {\large{$m_1$}};
        \node[draw=none, fill=none] at (0.4,  0.9) {\large{$c_1$}};

        \node[draw=none, fill=none] at (4.25, 2.2) {\large{$k_2$}};
        \node[draw=none, fill=none] at (6, 1) {\large{$m_2$}};
        \node[draw=none, fill=none] at (3.9, 0.9) {\large{$c_2$}};

\end{tikzpicture}

        \caption{}
        \label{fig:priguseni_ekvivalentni_sustav-2dof}
    \end{subfigure}
    \vfill
    \vspace{0.5cm}
    \begin{subfigure}[b]{1\textwidth}
        \centering
        \begin{tikzpicture}
	%uteg_1
	\draw[black,thick] (1.5, 2) rectangle (3.5, 0);

	%sile
	\draw[strelica1] (1.5, 1.75) -- (0.25, 1.75) 
		node[pos=1, above]{$k_1u_1$};
        \draw[strelica1] (1.5, 0.25) -- (0.25, 0.25)
                node[pos=1, above]{$c_1\dot{u}_1$};

	\draw[strelica1] (3.5, 1) -- ( 5.5, 1) 
		node[pos=1, right]{$p_1(t)$};
        \draw[strelica1] (3.5, 1.75) -- (5.5, 1.75);
%                node[pos=1, below]{$k_2(u_2-u_1)$};
        \draw[strelica1] (3.5, 0.25) -- (5.5, 0.25);
%                 node[pos=1, below]{$c_2(\dot{u}_2-\dot{u}_1)$}
                 
            
        %uteg_2
        \draw[black,thick] (7.5, 2) rectangle (9.5, 0);

        %sile
        \draw[strelica1] (7.5, 1.75) -- (6, 1.75)
                node[pos=1, above] {$c_2(\dot{u}_2-\dot{u}_1)$};
        \draw[strelica1] (7.5, 0.25) -- (6, 0.25)
                node[pos=1, below]{$k_2(u_2-u_1)$};
        \draw[strelica1] (9.5, 1) -- (10.5, 1)
                node[pos=1, above]{$p_2(t)$};

\end{tikzpicture}

        \caption{}
        \label{fig:priguseni_sustav_okvir_sile-2dof}
    \end{subfigure}
    \caption{Idealizirani sustav s dva stupnja slobode i prigušenjem:
    (\subref{fig:priguseni_sustav_okvir-2dof}) dvoetažni posmični okvir s prigušenjem;
    (\subref{fig:priguseni_ekvivalentni_sustav-2dof}) ekvivalentni prigušeni model;
    (\subref{fig:priguseni_sustav_okvir_sile-2dof}) prikaz sila; } 
    \label{fig:priguseni_sustav-2dof}
\end{figure}

Jednadžba gibanja sa slike \ref{fig:priguseni_sustav-2dof} glasi:
\begin{equation}\label{eq:JednadzbaGibanjaPriguseniMDF}
    \begin{dcases}
        m_1\ddot{u}_1+\dot{u}_1(c_1+c_2)-c_2\dot{u}_2+u_1(k_1+k_2)-k_2u_2=p_1(t)\\
        m_2\ddot{u}_2+c_2\dot{u}_2-c_2\dot{u}_1+k_2u_2-k_2u_1=p_2(t)
    \end{dcases}
\end{equation}

Sustav \eqref{eq:JednadzbaGibanjaPriguseniMDF} možemo zapisati u matričnom obliku:
\begin{equation}\label{eq:JednadzbaGibanjaPriguseniMDF-matrice}
    \begin{bmatrix}
        m_1 & 0 \\
        0   & m_2
    \end{bmatrix}
    \begin{Bmatrix}
        \ddot{u}_1\\
        \ddot{u}_2
    \end{Bmatrix}
    +
    \begin{bmatrix}
        c_1 + c_2 & -c_2\\
        -c_2 & c_2
    \end{bmatrix}
    +
    \begin{bmatrix}
        k_1+k_2 & -k_2\\
        -k_2 & k_2
    \end{bmatrix}
    \begin{Bmatrix}
        u_1\\
        u_2
    \end{Bmatrix}
    =
    \begin{Bmatrix}
        p_1\\
        p_2 
    \end{Bmatrix}
\end{equation}

Kompaktniji zapis jednadžbe pod \eqref{eq:JednadzbaGibanjaPriguseniMDF-matrice} 
prikazan je u nastavku:
\begin{equation}\label{eq:JednadzbaGibanjaPrigusenoMDF-matricni}
    \mm\vtor{u}{:}+\cc\vtor{u}{.}+\kk\vtor{u}{} = \vtor{p}{}
\end{equation}

Uvodi se $u=\ppsi\vtor{q}{}$ te jednadzba \eqref{eq:JednadzbaGibanjaPrigusenoMDF-matricni}
poprima slijedeći oblik:
\begin{equation}
    \mm\ppsi\vtor{q}{:}+\cc\ppsi\vtor{q}{.}+\kk\ppsi\vtor{q}{}=\vtor{p(t)}{}\qquad 
\end{equation}

Množenjem s $\ppsi^T$ dobijemo:
\begin{equation}\label{eq:JednadzbaGibanjaPrigusenoMDF-svedi}
    \ppsi^T\mm\ppsi\vtor{q}{:}+\ppsi^T\cc\ppsi\vtor{q}{.}+\ppsi^T\kk\ppsi\vtor{q}{}=\ppsi^T\vtor{p(t)}{}\\
\end{equation}

Iz poglavlja \ref{sec:normiranje} poznato je da vrijedi:
\begin{align*}
    \M &= \ppsi^T\mm\ppsi\\
    \K &= \ppsi^T\kk\ppsi
\end{align*}
Gdje su matrice $\M$ i $\K$ dijagonalne matrice modalne mase odnosno krutosti. Osim
matrica $\M$ i $\K$ uvodi se i matrica $\C$ koju nazivamo matricom modalnog
prigušenja, pa je jednadžbu \eqref{eq:JednadzbaGibanjaPrigusenoMDF-svedi} moguće
zapisati na slijedeći način:

\begin{equation}\label{eq:ModalneJednadzbe}
    \M\vtor{q}{:} + \C\vtor{q}{.} + \K\vtor{q}{} = \vtor{P}{}
\end{equation}

Gdje je $\vtor{P}{}$ vektor modalnog opterećenja. Matrica modalnog prigušenja $\C$ može 
i ne mora biti dijagonalna pa je potrebno razmotriti dva različita slučaja.
    \begin{enumerate}
        \item Ukoliko je matrica $\C$ dijagonalna, jednadžba
            \eqref{eq:ModalneJednadzbe} predstavlja skup međusobno neovisnih
            jednadžbi pa je na takvome sustavu primjenjiva klasična modalna analiza.
            Takav oblik prigušenja naziva se \textit{klasični oblik prigušenja}.
            
        \item Ukoliko matrica $\C$ nije dijagonalna, jednadžba
            \eqref{eq:ModalneJednadzbe} predstavlja sustav međusobno povezanih
            diferencijalnih jednadžbi, te na takav sustav nije primjenjiva klasična
            modalna analiza. Takav oblik prigušenja naziva se \textit{općim oblikom
            prigušenja}.
    \end{enumerate}

Matrica modalnog prigušenja zadana je slijedećim izrazom:
\begin{equation}\label{eq:ModalnoPrigusenje}
    C = \ppsi^T \cc \ppsi
\end{equation}

Za sustav s dva stupnja slobode, matrica $\C$ glasi:
\begin{equation}
    \begin{bmatrix}
        I & II\\
        III & IV
    \end{bmatrix}
\end{equation}

Gdje je:
\begin{align*}
    I  &= \psi_{1,1}^2c_1+\psi_{1,1}^2c_2+\psi_{2,1}c_2\psi_{1,1}+\psi_{2,1}^2c_2-\psi_{1,1}c_2\psi_{2,1}\\
    II &= \psi_{1,1}c_1\psi_{1,2}+\psi_{1,1}c_2\psi_{1,2}-\psi_{2,1}c_2\psi_{1,2}+\psi_{2,1}c_2\psi_{2,2}-\psi_{1,1}c_2\psi_{2,2}\\
    III&= \psi_{1,1}c_1\psi_{1,2}+\psi_{1,1}c_2\psi_{1,2}-\psi_{2,1}c_2\psi_{1,2}+\psi_{2,1}c_2\psi_{2,2}-\psi_{1,1}c_2\psi_{2,2}\\
    IV &= \psi_{1,2}^2c_1+\psi_{1,2}^2c_2i-\psi_{2,2}c_2\psi_{1,2}+\psi_{2,2}^2c_2-\psi_{1,2}c_2\psi_{2,2}
\end{align*}

Matrica $\C$ biti će dijagonalna ukoliko su elementi iznad i ispod glavne dijagonale
jednaki nula, što znači da je potrebno riješiti slijedeći sustav jednadžbi:
\begin{equation}
    \begin{dcases}
        II=0\\
        III=0
    \end{dcases}
\end{equation}

Zbog toga što je matrica $\C$ simetrična s obzirom na glavnu dijagonalu (u ovom
slučaju), preostaje jedna jednadžba s dvije nepoznanice $c_1$ i $c_2$, odnosno:
\begin{equation}\label{eq:JednadzbaRazdiobaPrigusenja}
    \psi_{1,1}c_1\psi_{1,2}+\psi_{1,1}c_2\psi_{1,2}-\psi_{2,1}c_2\psi_{1,2}+\psi_{2,1}c_2\psi{2,2}-\psi_{1,1}c_2\psi_{2,2}=0
\end{equation}

Navedena jednadžba ima beskonačno mnogo rješenja slijedećeg oblika:
\begin{equation}\label{eq:JednadzbaRazdiobaPrigusenja-rjesenje}
    \frac{c_1}{c_2} = -\frac{\psi_{1,2}\psi_{1,1}-\psi_{2,2}\psi_{1,1}+\psi_{2,2}\psi_{2,1}-\psi_{1,2}\psi_{2,1}}
                            {\psi_{1,2}\psi_{1,1}}
\end{equation}

Rješenja jednadžbe \eqref{eq:JednadzbaRazdiobaPrigusenja}
predstavljaju svaki $c_1$ i $c_2$, čiji je međusobni omjer jednak razlomku s desne
strane izraza \eqref{eq:JednadzbaRazdiobaPrigusenja-rjesenje}. Iz jednadžbi
\eqref{eq:JednadzbaRazdiobaPrigusenja} i \eqref{eq:JednadzbaRazdiobaPrigusenja-rjesenje} 
slijedi da će matrica $\C$ biti dijagonalna samo ako omjer koeficijenata prigušenja $c_1/c_2$ 
zadovoljava jednakost pod \eqref{eq:JednadzbaRazdiobaPrigusenja-rjesenje}. U
protivnom, matrica $\C$ nije dijagonalna a u sustavu vlada opći oblik prigušenja.
Omjer prigušenja nazivamo \textit{razdiobom prigušenja u sustavu}. Zaključno, oblik
matrice $\C$ ovisi o razdiobi prigušenja u sustavu.
\par

Bitno je za napomenuti da u slučaju općeg prigušenja, oblici titranja sustava
su različiti od oblika titranja neprigušenog sustava, a iz jednadžbe pod
\eqref{eq:JednadzbaGibanjaPriguseniMDF-matrice} dobije se kompleksni problem
vlastitih vrijednosti čija rješenja su kompleksni vlastiti vektori $\psi$ i
kompleksne vlastite vrijednosti $\omega$. Zbog svoje dugotrajnosti i matemtičke složenosti,
navedeni slučaj neće biti razmatran u ovome radu.
\par

Dakle, jednadžbu \eqref{eq:ModalneJednadzbe} moguće je riješiti ukoliko je sustav
prigušen klasičnim oblikom prigušenja (matrica $\C$ je dijagonalna), pri čemu je
postupak analogan postupku iz poglavlja ~\ref{sec:modalnaAnalizaNepriguseniSustav}. Modalna
jednadžba n-tog oblika titranja glasi:
\begin{equation}\label{eq:ModalnaJednadzba_n}
    M_n\ddot{q}_n + C_n\dot{q}_n  + K_nq = P_n(t)
\end{equation}
Gdje je $M_n$, $C_n$, $K_n$ i $P_n$ modalna masa, modalno prigušenje, modalna
krutost i modalno opterećenje n-tog oblika titranja.
Dijeljenjem jednadžbe \eqref{eq:ModalnaJednadzba_n} s modalnom masom $M_n$ dobijemo:
\begin{equation}\label{eq:modalnaPriguseno-konacno}
    \ddot{q}_n + 2\omega_n\zeta_n\ddot{q}_n + \omega_n^2q = \frac{P_n}{M_n}
\end{equation}

Gdje je $\zeta_n$ relativni faktor prigušenja n-tog oblika titranja a $\omega_n$
prirodna frekvencija n-tog oblika titranja. Forma rješenja diferencijalne jednadžbe 
\eqref{eq:modalnaPriguseno-konacno} prikazana je pod \eqref{eq:modalnaPriguseno-rjesenje} 
a predstavlja odziv n-tog oblika titranja sustava s više stupnjeva slobode. 

\begin{equation}\label{eq:modalnaPriguseno-rjesenje}
    u_n = \psi_n q_n(t)
\end{equation}

Ukupni odziv dobijemo superpozcijom odziva svih modalnih oblika, odnosno:
\begin{equation} 
    u(t) = \sum_{n=1}^Nu_n(t) = \sum_{n=1}^N\psi_nq_n(t)
\end{equation}
