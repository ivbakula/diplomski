Razvojem računala i računalne tehnologije otvara se prostor za razvitak dinamike
konstrukcija. Za razumijevanje osnovnih teoretskih principa i pojmova dinamike
konstrukcija potrebno je kvalitetno poznavanje najjednostavnijeg sustava, tj.
sustava s jednim stupnjem slobode. Osim didaktičke svrhe, sustav s jednim stupnjem
slobode može poslužiti i kao model za razvitak eksperimentalnih metoda kao što je
rezonancijski pokus. 
\par

U inženjerskoj praksi vrlo često se susreću sustavi s više stupnjeva slobode koji
mogu biti izrazito složeni. U graditeljstvu sustav s više stupnjeva slobode pojavljuje
se kod visokih građevina, primjerice nebodera. Visoke građevine posebno su osjetljive
na horizontalna dinamička opterećenja, poput vjetara i potresa. 
Poznavanjem dinamičkih parametara konstrukcije, moguće je efekte dinamičkog
opterećenja svesti na minimum odnosno učinkovito dimenzionirati konstrukciju na
dinamičko opterećenje. Primjerice, dinamički parametri konstrukcije mogu se 
iskoristiti za projektiranje posebnih prigušivača koji se zatim ugrađuju na pogodna 
mjesta u konstrukciju. Dinamički parametri konstrukcija najšešće se određuju
modalnom analizom.
\par

U rudarstvu postupci modalne analize primjenjivi su na analizu odziva konzole
rotornog bagera, u separacijskim postrojenjima na analizu vibracijskih sita i sl. 
\par

Osim u graditeljstvu i rudarstvu, postupci modalne analize koriste se u brojnim tehničkim i
znanstvenim disciplinama, kao što su strojarstvo, zrakoplovno inženjerstvo, autoindustrija
i sl.
\par

Jedna od meni osobno interesantnijih primjena modalne analize je u akustici. U
akustici, modalna analiza se može koristiti za određivanje projektnih parametara
zvučnika te za otkrivanje „malih tajni velikih majstora”, na primjer zašto
Stradivarijeva violina  zvuči bolje od prosječnih.

