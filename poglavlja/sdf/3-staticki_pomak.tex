\section{Statički pomak}
Zanemarivanjem ubrzanja u diferencijalnoj jednadžbi pod
\eqref{eq:jednadzba_gibanja_nepriguseni_nesredjeno} dobijemo:
\begin{equation}\label{eq:vremenska_funkcija_statickog_pomaka}
	u(t)=\frac{p_0}{k}sin(\omega t)
\end{equation}

Navedena jednadžba predstavlja \textit{vremensku funkciju statičkog pomaka} (~\cite{dk_skripta}). Pomak
nazivamo statičkim jer se zanemaruje dinamički utjecaj sile pobude (pretpostavlja se
spora promjena opterećenja).
Vremenska funkcija statičkog pomaka, preko Hookeovog zakona, stavlja u odnos silu 
pobude ($p_0sin(\omega t)$) i pomak sustava $u(t)$. Zanemarivanjem funkcije sinus, dobijemo
amplitudu statičkog pomaka koja je definirana slijedećom jednadžbom:
\begin{equation}
	(u_{st})_0 = \frac{p_0}{k}
\end{equation}


