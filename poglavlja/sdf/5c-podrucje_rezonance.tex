\subsection{Analiza područja rezonancije}
Osim rezonancijske frekvencije potrebno je odrediti i pojas polovice snage
(\textit{engl.} half-power bandwidth) vrha frekvencijske funkcije odziva, 
koji je prikazan na slici \ref{fig:hpb}. Pojas polovice snage vrha frekvencijske funkcije
odziva bitan je iz dva razloga:
\begin{enumerate}
    \item osim pobude rezonancijskom frekvencijom, opasne su i pobude frekvencijama iz
        njezinog okoliša. Navedeni okoliš definiran je pojasom polovice snage.
    \item Zbog praktične primjene - pojas polovice snage koristi se u pokusima
        za određivanje stupnja prigušenja konstrukcija
\end{enumerate}
\begin{figure}[H]
    \begin{tikzpicture}[scale=0.75]
    \begin{axis} [
        ylabel = Dinamički koeficijent $R_d$,
        xlabel = $\omega/\omega_n$,
        xmin = 0, xmax = 4,
        ymin = 0, ymax = 6,
        xtick = {0, 1, 2, 3, 4},
        ytick = {1, 2, 3, 4, 5, 6},
        grid=both, %minor tick num=0.5,
     ]

    \addplot [
        domain=0:4,
        samples=500,
        color=black,
        thick,
    ]{1/((1-x^2)^2+(2*0.1*x)^2)^0.5};

    \draw[color=black,pattern=north east lines, pattern color=black] (0.89,0) 
        rectangle (1.08,3.555);
    \draw[strelica1] (0.75, 1) -- (0.89, 1);
    \draw[strelica1] (2.4, 1) -- (1.08,1)
        node[above,pos=0.35] {\footnotesize{Pojas polovice snage = $2\zeta$}};

    %kotiraj rezonancijsku amplitudu
    \draw[<->,line width=1pt] (3.5,0) -- (3.5, 5.027)
        node[pos=0.5,above,rotate=90]{\footnotesize{Rezonancijska amplituda}};
    \draw(0.995,5.027) -- (3.5,5.027);

    \node at (0.81, 0.5) {$\frac{\omega_a}{\omega_n}$};
    \node at (1.16, 0.5) {$\frac{\omega_b}{\omega_n}$};

    %kotiraj pojas polovice snage
    \draw[<->,line width=1pt] (3,0) -- (3,3.555)
        node[pos=0.5,above,rotate=90]{\footnotesize{$1/\sqrt{2}$ rezonancijske amplitude}};
    \draw(1,3.555) -- (3,3.555);

    \end{axis}
\end{tikzpicture}

    \caption{Definicija pojasa polovice snage}
    \label{fig:hpb}
\end{figure}
Dinamički faktor pomaka $R_d$ za odziv dvostruko manje snage od odziva maksimalnog
dinamičkog faktora računa se prema sljedećoj relaciji:
\begin{equation}\label{eq:hpb}
    R_d = \frac{1}{\sqrt{2}}R_d^{max}
\end{equation}

Sa slike \ref{fig:hpb} je vidljivo da je dinamički faktor $R_d$ iz \eqref{eq:hpb} definiran za
dvije vrijednosti frekvencije pobude: $\omega_a$ i $\omega_b$. Raspisivanjem
jednadžbe pod \eqref{eq:hpb} dobijemo:

\begin{equation}\label{eq:hpb_izvod_1}
        \frac{1}{\sqrt{(1-(\omega/\omega_n)^2)^2+(2\zeta\omega/\omega_n)^2}}=
            \frac{1}{\sqrt{2}}\frac{1}{2\zeta\sqrt{1-\zeta^2}}\\
\end{equation}

Kvadriranjem \eqref{eq:hpb_izvod_1} dobijemo:
\begin{equation}\label{eq:hpb_izvod_2}
    \left(1-\left(\frac{\omega}{\omega_n}\right)^2\right)^2
    +\left(2\zeta\frac{\omega}{\omega_n}\right)^2 =
    8\zeta^2(1-\zeta^2)
\end{equation}

Raspisivanjem i grupiranjem po $\omega/\omega_n$ dobijemo:
\begin{equation}\label{eq:hpb_izvod_3}
    \left(\frac{\omega}{\omega_n}\right)^4
    -2(1-2\zeta^2)\left(\frac{\omega}{\omega_n}\right)^2
    +1-8\zeta^2(1-\zeta^2)=0
\end{equation}

Izraz \eqref{eq:hpb_izvod_3} je kvadratna jednadžba, a njezinim rješavanjem (po
$\omega/\omega_n$) dobijemo:
\begin{equation}\label{eq:hpb_kvadratna_1}
    \left(\frac{\omega}{\omega_n}\right)^2 = 
        (1-2\zeta^2)\pm 2\zeta\sqrt{1-\zeta^2}
\end{equation}

Za $\zeta^2 \approx 0$ izraz \eqref{eq:hpb_kvadratna_1} postaje:
\begin{equation}
    \left(\frac{\omega}{\omega_n}\right)^2 \approx
        1 \pm 2\zeta
\end{equation}

Odnosno:
\begin{equation}\label{eq:hpb_kvadratna_2}
    \left(\frac{\omega}{\omega_n}\right)\approx
        \sqrt{1 \pm 2\zeta}
\end{equation}
Jednadžba \eqref{eq:hpb_kvadratna_2}, nakon aproksimacije korijena s prva dva člana Taylorovog
reda glasi:
\begin{equation}\label{eq:hpb_kvadratna_konacno}
    \frac{\omega}{\omega_n} = 1 \pm \zeta
\end{equation}

Frekvencije $\omega_a$ i $\omega_b$ dobiju se iz \eqref{eq:hpb_kvadratna_konacno}:
\begin{align}
    \omega_a &= (1-\zeta)\omega_n\\
    \omega_b &= (1+\zeta)\omega_n\\
\end{align}

Oduzimanjem $\omega_b-\omega_a$ dobijemo:
\begin{equation}\label{eq:hpb_zeta_omega}
        2\zeta = \frac{\omega_b-\omega_a}{\omega_n}
\end{equation}

Te konačno, dijeljenjem brojnika i nazivnika s $2\pi$:
\begin{equation}\label{eq:hpb_zeta_f}
    \zeta\approx\frac{f_b-f_a}{2f_n}
\end{equation}

gdje je $f=\omega/2\pi$ kružna frekvencija. Jednadžbe pod \eqref{eq:hpb_zeta_omega}
i \eqref{eq:hpb_zeta_f} bitne su jer omogućuju određivanje koeficijenta relativnog prigušenja
$\zeta$ bez potrebe za poznavanjem intenziteta sile pobude (~\cite{chopra2011}). 


