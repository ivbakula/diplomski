\subsection{Zapis prisilnog dijela odziva}\label{ss:Prisilni_standardno}
Prisilni dio odziva definiran je jednadžbom prikazanom u nastavku:
\begin{equation}\label{eq:samo_prisilni}
    u(t)=C\sin(\omega t) + D\cos(\omega t)
\end{equation}
pri čemu su konstante C i D izračunate pod \eqref{eq:np_koef_C} i \eqref{eq:np_koef_D}.
Prisilni odziv pod \eqref{eq:samo_prisilni} možemo zapisati u obliku $u_0\sin(\omega t
- \phi)$ 
korištenjem slijedećeg trigonometrijskog identiteta (~\cite{dk_skripta}):
\begin{equation}\label{eq:prisilni_dio_odziva}
    u_0\sin(\omega t - \phi) = C\sin(\omega t) + D\cos(\omega t)
\end{equation}

gdje je:
\begin{alignat}{2}
    &\text{amplituda dinamičkog pomaka} \quad & u_0 &= \sqrt{C^2+D^2}\label{eq:u_0-izvod-prvo}\\
    &\text{kašnjenje u fazi} & \phi &= \arctan{-\frac{D}{C}}\label{eq:fi-izvod}
\end{alignat}

Raspisivanjem formule pod \eqref{eq:u_0-izvod-prvo} dobijemo:
\begin{equation}\label{eq:u_0-izvod-raspisano}
    u_0 = \frac{p_0}{k}\frac{1}{\sqrt{(1-(\omega/\omega_n)^2)^2+(2\zeta\omega/\omega_n)^2}}
        = \frac{(u_{st})_0}{\sqrt{(1-(\omega/\omega_n)^2)^2+(2\zeta\omega/\omega_n)^2}}
\end{equation}

Definiramo dinamički koeficijent pomaka ili koeficijent povećanja pomaka ($R_d$) kao omjer
amplitude dinamičkog i statičkog pomaka.
\begin{equation}\label{eq:R_d-izvod-konacno}
    R_d = \frac{u_0}{(u_{st})_0}=\frac{1}{\sqrt{(1-(\omega/\omega_n)^2)^2+(2\zeta\omega/\omega_n)^2}}
\end{equation}

Fazni kut $\phi$ dobijemo raspisivanjem izraza pod \eqref{eq:fi-izvod} te dobijemo:
\begin{equation}\label{eq:fi-izvod-konacno}
    \phi = \arctan{\frac{2\zeta(\omega/\omega_n)}{1-(\omega/\omega_n)^2}}
\end{equation}

Te konačno, uvrštavanjem \eqref{eq:R_d-izvod-konacno} i \eqref{eq:fi-izvod-konacno} u
\eqref{eq:prisilni_dio_odziva} dobijemo prisilni odziv koji glasi:
\begin{equation}
    u(t) = \frac{p_0}{k}R_d\sin(\omega t - \phi) = (u_{st})_0R_d\sin(\omega t - \phi)
\end{equation}

\subsubsection{Prisilni odziv kao poseban slučaj rješenja $p(t)=p_oe^{\alpha
t}\sin(\omega t)$}\footnotemark \footnotetext{Postupak je temeljen na (~\cite{sisbabic}) str 112}
Odziv na pobudu sinusnom silom može se odrediti konvolucijom prijenosne funkcije i
funkcije pobude u vremenskoj domeni. Zadana je pobuda sinusnom silom oblika:
\begin{equation}\label{eq:pobuda_sinusnom_silom}
    p(t)=p_0e^{\alpha t} \sin(\omega t)
\end{equation}

Jednadžbu \eqref{eq:pobuda_sinusnom_silom} možemo zapisati u eksponencijalnom
obliku:
\begin{equation}\label{eq:pobuda_sinusnom_silom_eksponencijalni}
    p(t)=\frac{1}{2i}p_0(e^{(\alpha+i\omega)t}-e^{(\alpha-i\omega)t})
\end{equation}

Uvodimo:
\begin{align}
    s  &= \alpha + i\omega \label{eq:s_normalno}\\
    s^* &= \alpha - i\omega \label{eq:s_konjugirano}
\end{align}
Uvrštavanjem \eqref{eq:s_normalno} i \eqref{eq:s_konjugirano} u
\eqref{eq:pobuda_sinusnom_silom_eksponencijalni} dobijemo:
\begin{equation}\label{eq:pobuda_sinusnom_silom_konacno}
    p(t) = \frac{1}{2i}p_0(e^{st}-e^{s^*t})
\end{equation}

Prijenosna funkcija sustava (s izlučenim članom $1/k$) zadana je u obliku $1/k
h(t)$. Izračun odziva je prikazan u nastavku:
\[
    u(t) = \frac{1}{k}(h*p)(t) 
         = \frac{1}{k}\int_0^\infty h(t-\tau)p(\tau)d\tau
\]

Uvodi se supstitucija $\lambda=t-\tau$.
\[
    u(t) = \frac{1}{k}\int_0^\infty h(\lambda)p(t-\lambda)d\lambda
         = \frac{1}{k}\int_0^\infty h(\lambda)\left[
             \frac{1}{2i}p_0\left(e^{st-s\lambda}-e^{s^*t-s^*\lambda}\right)
             \right]d\lambda
\]
Raspisivanjem i sređivanjem dobijemo:
\begin{equation}\label{eq:prisilni_transformacija_prijenosne}
    u(t) = \frac{1}{2i}\frac{p_0}{k} \left[
                e^{st}
                \underbrace{
                    \int_0^\infty h(\lambda)e^{-s\lambda}d\lambda
                }_{\text{\textbf{$I)$}}}
                -e^{s^*t}
                \underbrace{
                    \int_0^\infty h(\lambda)e^{-s^*\lambda}d\lambda
                }_{\text{\textbf{$II)$}}}
        \right]
\end{equation}
Uočimo da integrali označeni s $I)\text{ i } II)$ predstavljaju Laplaceovu
transformaciju prijenosne funkcije sustava pa izraz pod \eqref{eq:prisilni_transformacija_prijenosne}
postaje:
\begin{equation}\label{eq:prisilni_transformacija_prijenosne_kk}
    u(t)=\frac{1}{2i}\frac{p_0}{k}(e^{st}H(s)-e^{s^*t}H(s))
\end{equation}

Varijabla $s^*$ je kompleksno konjugirana varijabla $s$, pa je funkcija
$H(s^*)$ kompleksno konjugirana funkcija $H(s)$, odnosno:
\begin{equation}\label{eq:prijenosne_odnos}
    H(s^*)=H^*(s) 
\end{equation}

Korištenjem \eqref{eq:prijenosne_odnos} jednadžba \eqref{eq:prisilni_transformacija_prijenosne_kk}
postaje:
\begin{equation}\label{eq:prisilni_transformacija_prijenosne_hk}
    u(t)=\frac{1}{2i}\frac{p_0}{k} (e^{st}H(s)-e^{s^*t}H^*(s))
\end{equation}

Poseban slučaj je za $\alpha = 0$ odnoso $s=i\omega \text{ i } s^*=-i\omega$. Tada
prijenosne funckije sustava postaju frekvencijske funkcije odziva, a izraz pod
\eqref{eq:prisilni_transformacija_prijenosne_hk} glasi:

\begin{equation}\label{eq:prisilni_transformacija_frf}
    u(t) = \frac{1}{2i}\frac{p_0}{k}(e^{i\omega t}H(\omega i) - e^{-i\omega t}H^*(i\omega))
\end{equation}

Frekvencijska funkcija odziva $H(\omega i)$ zadana je jednadžbom \eqref{eq:frf_pravokutni}.
Možemo uočiti da je njezin imaginarni dio negativan, što znači da je imaginarni dio
funkcije $H^*(\omega i)$ pozitivan.
\begin{figure}[H]
    \input{./skice/sdf/frf-gauss}
    \caption{Shematski prikaz funkcija $H(\omega i) \text{ i }H^*(\omega i)$ u
    Gaussovoj ravnini}
    \label{fig:frf-gauss}
\end{figure}

Modul funkija $H(\omega)$ i $H^*(\omega)$ je isti a definiran je jednadžbom
\eqref{eq:magnitudniSpektar} a fazni kut jednadžbom \eqref{eq:fazniSpektar}. Sa
slike \ref{fig:frf-gauss} vidljivo je da je kut $\phi$ što ga zatvara $H(\omega)$ s
realnom osi negativan, a kut što ga zatvara $H^*(\omega)$ s realnom osi pozitivan.
Trigonometrijski zapisi funkcija $H(\omega) \text{ i } H^*(\omega)$ dati su u
nastavku.
\begin{align}
    H(\omega) &= |H(\omega)|e^{-i\phi} \label{eq:trig_zapis}\\
    H^*(\omega) &= |H(\omega)|e^{i\phi} \label{eq:trig_zapis_hk} %hk- h-kompleksno konjugiran
\end{align}

Uvrštavanjem \eqref{eq:trig_zapis} i \eqref{eq:trig_zapis_hk} u \eqref{eq:prisilni_transformacija_frf}
dobijemo:
\begin{equation}
    u(t)=\frac{1}{2i}\frac{p_0}{k}|H(\omega)|(e^{i\omega t}e^{-i\phi}-e^{-i\omega t}e^{i\phi})
        =\frac{1}{2i}\frac{p_0}{k}|H(\omega)|(2i\sin(\omega t -\phi))
\end{equation}

Te konačno:
\begin{equation}\label{eq:prisilni_alternativno_rjesenje}
    u(t)=\frac{p_0}{k}|H(\omega)|\sin(\omega t - \phi)
\end{equation}
Jednadžba \eqref{eq:prisilni_alternativno_rjesenje} predstavlja prisilni dio odziva.

Iz \eqref{eq:R_d-izvod-konacno} i \eqref{eq:magnitudniSpektar} slijedi da je
\textit{dinamički faktor} zapravo \textit{norma (ili intenzitet)} frekvencijske 
funkcije odziva.  Stoga jednadžbu pod \eqref{eq:prisilni_alternativno_rjesenje} 
možemo zapisati kao:
\begin{equation}\label{eq:prisilni_alternativno_rjesenje_Rd}
    u(t)=\frac{p_0}{k}R_d\sin(\omega t - \phi)
\end{equation}

Iz \eqref{eq:prisilni_alternativno_rjesenje_Rd} možemo definirati fizikalnu 
interpretaciju frekvencijske funkcije odziva.  Dakle, frekvencijska funkcija 
odziva definira odnos između pobude i odziva. Taj odnos je kompleksan jer je 
opisan intenzitetom, odnosno dinamičkim koeficijentom pomaka, te faznim kutom 
kompleksne funkcije definirane u $s$ domeni (~\cite{koscakturkalj}).  Drugim riječima, 
frekvencijska funkcija odziva nas upućuje na slijedeća svojstva (prisilnog) odziva:
\begin{enumerate}
    \item Frekvencija odziva biti će jednaka frekvenciji pobude ($\omega$).
    \item Amplituda odziva biti će skalirana amplituda statičkog pomaka. Prisjetimo
        se da je amplituda statičkog pomaka u izravnoj vezi sa amplitudom sile pobude.
    \item Odziv će zaostajati u fazi za pobudom. 
\end{enumerate}

Skaliranje amplitude statičkog pomaka $(u_{st})_0$ definirano je \textit{dinamičkim
koeficijentom} $R_d$. Navedeni koeficijent je konkretna vrijednost frekvencijske funkcije
odziva za određenu frekvenciju $\omega$. Analogno tome, kašnjenje u fazi definirano je 
faznim kutom funkcije odziva.

