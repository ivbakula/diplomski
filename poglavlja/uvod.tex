U ovom radu razmotren je dinamički sustav pod utjecajem harmonijske pobude. Za
matematički model dinamičkog sustava uzet je linearni sustav. Linearni sustav je
ulazno-izlazni model što znači da su pobuda i odziv u uzročno posljedičnoj vezi.
Linearne sustave je moguće razmatrati kao „crne kutije”. Primjerice, na slici je
zadan linearni sustav $A$ na koji djeluju pobude $f(t)$ i $g(t)$.


Sa slika su uočljive slijedeće značajke linearnog sustava: 
\begin{enumerate}
    \item sinusna pobuda rezultira sinusnim odzivom.
    \item amplituda odziva je skalirana amplituda pobude. 
    \item odziv je pomaknut u fazi u odnosu na pobudu.  
    \item odziv na složenu harmonijsku pobudu, može se zapisati kao linearna
    kombinacija jednostavnih harmonijskih pobuda (princip superpozicije)
\end{enumerate}


Posljedica značajki 1) - 3) jest mogućnost određivanja odziva ukoliko se poznaje
faktor skaliranja i fazno kašnjenje. Fakotr skaliranja i fazno kašnjenje određuju se
frekvencijskim funkcijama odziva koje su određene u poglavlju \ref{sec:frf}. 
